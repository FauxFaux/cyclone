\subsection{The compiler}
\subsubsection*{General options}
The Cyclone compiler has the following command-line options:
\begin{description}
\item[-help] Print a short description of the command-line options.
\item[-v] Print compilation stages verbosely.
\item[--version] Print version number and exit.
\item[-o \textit{file}] Set the output file name to \textit{file}.
\item[-D\textit{name}]
  Define a macro named \textit{name} for preprocessing.
\item[-D\textit{name}=\textit{defn}]
  Give macro \textit{name} the definition \textit{defn} in preprocessing.
\item[-B\textit{dir}]
  Add \textit{dir} to the list of directories to search for special
  compiler files.
\item[-I\textit{dir}]
  Add \textit{dir} to the list of directories to search for include files.
\item[-L\textit{dir}]
  Add \textit{dir} to the list of directories to search for libraries.
\item[-l\textit{lib}]
  Link library \textit{lib} into the final executable.
\item[-c]
  Produce an object (\texttt{.o}) file instead of an executable; do
  not link.
\item[-s]
  Remove all symbol table and relocation information from the executable.
\item[-O]
  Optimize.
\item[-O2]
  A higher level of optimization.
\item[-O3]
  Even more optimization.
\item[-p]
  Compile for profiling with the \texttt{prof} tool.
\item[-pg]
  Compile for profiling with the \texttt{gprof} tool.
\item[-pa]
  Compile for profiling with the \texttt{aprof} tool.
\item[-M]
  Produce dependencies for inclusion in a makefile.
\item[-MG]
  When producing dependencies assume missing files are generated.
  Must be used with \texttt{-M}.
\item[-MT \textit{file}]
  Make \textit{file} be the target of any dependencies generated using
  the \texttt{-M} flag.
\item[-E]
  Stop after preprocessing.
\item[-S]
  Stop after producing assembly code.
\item[-nogc]
  Don't link in the garbage collector.
\end{description}

\subsubsection*{Developer options}

In addition, the compiler has some options that are primarily of use
to its developers:
\begin{description}
\item[-g]
  Compile for debugging.  This is currently only useful for compiler
  developers, as the debugging information reflects the C code that
  the Cyclone code is compiled to, and not the Cyclone code itself.
\item[-stopafter-parse]
  Stop after parsing.
\item[-stopafter-tc]
  Stop after type checking.
\item[-stopafter-toc]
  Stop after translation to C\@.
\item[-ic]
  Activate the link-checker.
\item[-pp]
  Pretty print.
\item[-up]
  Ugly print.
\item[-tovc]
  Avoid gcc extensions in the C output.
\item[-save-temps]
  Don't delete temporary files.
\item[-save-c]
  Don't delete temporary C files.
\item[-use-cpp\textit{path}]
  Indicate which preprocessor to use.
\item[-nocyc]
  Don't add the implicit namespace Cyc to variable names in the C output.
\item[-noremoveunused]
  Don't remove externed variables that aren't used.
\item[-noexpandtypedefs]
  Don't expand typedefs in pretty printing.
\item[-printalltvars]
  Print all type variables (even implicit default effects).
\item[-printallkinds]
  Always print kinds of type variables.
\item[-printfullevars]
  Print full information for evars (type debugging).
\end{description}

% Local Variables:
% TeX-master: "main-screen"
% End:
