\section{Introduction}
Cyclone is a language for C programmers who want to write secure,
robust programs.  It's a dialect of C designed to be \emph{safe}: free
of crashes, buffer overflows, format string attacks, and so on.
Careful C programmers can produce safe C programs, but, in practice,
many C programs are unsafe.  Our goal is to make \emph{all} Cyclone
programs safe, regardless of how carefully they were written.  All
Cyclone programs must pass a combination of compile-time, link-time,
and run-time checks designed to ensure safety.

There are other safe programming languages, including Java, ML, and
Scheme.  Cyclone is novel because its syntax, types, and semantics are
based closely on C\@.  This makes it easier to interface Cyclone with
legacy C code, or port C programs to Cyclone.  And writing a new
program in Cyclone ``feels'' like programming in C: Cyclone gives
programmers the same control over data representations, memory
management, and performance that C has.

Cyclone's combination of performance, control, and safety make it a
good language for writing systems and security software.  We also
expect that the experience of writing such software in Cyclone will
motivate new research into safe, low-level languages.  For instance,
originally, all heap-allocated data in Cyclone was reclaimed via a
conservative garbage collector.  Though the garbage collector ensures
safety by preventing programs from accessing deallocated objects, it
also kept Cyclone from being used in latency-critical or
space-sensitive applications such as network protocols or device
drivers.  To address this shortcoming, we have added a region-based
memory management system based on the work of Tofte and Talpin.  The
region-based memory manager allows you some real-time control over
memory management and can significantly reduce space overheads when
compared to a conventional garbage collector.  Furthermore, the region
type system ensures the same safety properties as a collector: objects
cannot be accessed outside of their lifetimes.

You might wonder what is the difference between the Cyclone project
and a few other projects, notably
\href{http://lclint.cs.virginia.edu/}{LCLint} out of MIT and Virginia,
the Extended Static Checking (ESC) project out of Compaq SRC, and the
\href{http://www.cs.wisc.edu/~austin/scc.html}{Safe-C} project from
the University of Wisconsin.  These are good systems and we have great
respect for the work that has gone into them.  But their goals are
quite different from Cyclone and we view our work as somewhat
overlapping, but mostly complimentary.

Though we are interested in finding and eliminating bugs, we also want
to guarantee a certain level of safety. LCLint can't provide such a
guarantee, because the underlying language (C) and associated static
analyses are unsound (i.e., some bugs will not be caught). The reason
for this is that without changing C, it is extremely difficult to
guarantee safety without a lot of extra runtime overhead, or a lot of
false positives in your static analysis. LCLint is aimed at being a
practical tool for debugging C code, and as such, makes compromises to
eliminate false positives, but at the expense of soundness. Thus,
LCLint can't be used in settings where security is a concern.

ESC is similar in some ways to LCLint, but is built on top of Java. (A
previous version was built on top of Modula-3). Since Java and
Modula-3 are type-safe, there is a fundamental safety guarantee
provided by the ESC system. And in addition, one gets the extra
(really amazing) bug-finding power of ESC\@. However, as we argued
above, many systems programming tasks cannot be coded in languages
such as Java (or Modula-3) or at least can't be coded efficiently. In
addition, it is difficult to move legacy code written in C to these
environments.

Consequently, we see the ESC and LCLint projects as complimentary to
the Cyclone project. Ideally, after Cyclone reaches a stable design
point, we should be able to put an ``ESC/Cyclone'' or
``LCLint/Cyclone'' static debugging tool together.

The Safe-C project out of Wisconsin is perhaps the closest in spirit
to ours, in that they are attempting to provide a memory-safety
guarantee. However, the approach taken by Safe-C is quite different
than Cyclone: Programs continue to be written in C, but extra
indirections and dynamic checks are inserted to ensure that
abstractions are respected. For example, pointers in Safe-C are
actually rather complex data structures, and dereferencing a pointer
is an expensive operation as a number of checks must be done to ensure
that the pointer is still valid. In order to avoid some of these
overheads, the Wisconsin folks apply sophisticated, automatic
analyses.

The advantage of Safe-C is that, in principle, programs can becompiled
and run with no modification by the programmer. The disadvantages are
that (a) there is still a great deal of space and time overhead for
programs, (b) the data representations change radically. As such, the
whole system (e.g., the standard libraries) has to be recompiled and
pieces of code cannot be migrated from C to Safe-C\@. In contrast,
Cyclone goes to great lengths to preserve the data representations so
that Cyclone and C code can smoothly interoperate. Furthermore, many
more checks are performed statically in Cyclone. The downside is that
programmers have to change the code (typically, by modifying or adding
more typing information), and they can't always write what they want.
Our research is aimed at eliminating these drawbacks and to compare
against the Safe-C approach.

Obviously, Cyclone is a work in progress. The people involved now are
at Cornell and AT\&T\@.  In particular, Trevor Jim and I have worked out
most of the initial design and implementation.  Dan Grossman is
working on version 2.0 of the system which should add support for a
number of other features and clean up the type-checker considerably.
Mathieu Baudet has developed the link-checker.  James Cheney, Fred
Smith, Dave Walker, and Stephanie Weirich have contributed greatly
through discussions, if not code.  Jeff Vinocur, Nathan Lutchansky,
Matthew Harris, Kate Oliver, and Hubert Chao have also made many
valuable contributions.  Many of the ideas contained within Cyclone
are derived from our previous experience and work on the Popcorn
safe-C language and the Typed Assembly Language project.

% Local Variables:
% TeX-master: "main"
% End:
