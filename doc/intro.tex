\section{Introduction}

Cyclone is a language for C programmers who want to write secure,
robust programs.  It's a dialect of C designed to be \emph{safe}: free
of crashes, buffer overflows, format string attacks, and so on.
Careful C programmers can produce safe C programs, but, in practice,
many C programs are unsafe.  Our goal is to make \emph{all} Cyclone
programs safe, regardless of how carefully they were written.  All
Cyclone programs must pass a combination of compile-time, link-time,
and run-time checks designed to ensure safety.

There are other safe programming languages, including Java, ML, and
Scheme.  Cyclone is novel because its syntax, types, and semantics are
based closely on C\@.  This makes it easier to interface Cyclone with
legacy C code, or port C programs to Cyclone.  And writing a new
program in Cyclone ``feels'' like programming in C: Cyclone tries to
give programmers the same control over data representations, memory
management, and performance that C has.

Cyclone's combination of performance, control, and safety make it a
good language for writing systems and security software.  Writing such
software in Cyclone will, in turn, motivate new research into safe,
low-level languages.  For instance, originally, all heap-allocated
data in Cyclone were reclaimed via a conservative garbage collector.
Though the garbage collector ensures safety by preventing programs
from accessing deallocated objects, it also kept Cyclone from being
used in latency-critical or space-sensitive applications such as
network protocols or device drivers.  To address this shortcoming, we
have added a region-based memory management system based on the work
of Tofte and Talpin.  The region-based memory manager allows you some
real-time control over memory management and can significantly reduce
space overheads when compared to a conventional garbage collector.
Furthermore, the region type system ensures the same safety properties
as a collector: objects cannot be accessed outside of their lifetimes.

This manual is meant to provide an informal introduction to Cyclone.
We have tried to write the manual from the perspective of a C 
programmer who wishes either to port code from C to Cyclone, or
develop a new system using Cyclone.  Therefore, we assume a fairly
complete understanding of C.

Obviously, Cyclone is a work in progress and we expect to make
substantial changes to the design and implementation.  Your feedback
(and patience) is greatly appreciated.

\subsection{Acknowledgements}

The people involved in the development of Cyclone are at Cornell and
AT\&T\@.  Dan Grossman, Trevor Jim, and Greg Morrisett worked out the
initial design and implementation, basing the language to some degree
on Popcorn, a safe-C-like language that was developed at Cornell as
part of the
\href{http://www.cs.cornell.edu/talc}{Typed Assembly Language} (TAL)
project.  Mathieu Baudet contributed the bulk of the code for the
link-checker.  Matthew Harris did much of the hard work needed to
wrap and import the necessary libraries.  Yanling Wang ported bison and
flex to Cyclone.  Mike Hicks ported a number of libraries and programs
to Cyclone, helped with the configuration and installation procedures,
and many more things.  All of these people have also contributed by finding
and fixing various bugs.  James Cheney has added support for 
representation types, singleton ints, marshalling support, etc.
A number of other people have also helped
to find bugs and/or contributed key design ideas including 
Fred Smith, Nathan Lutchansky, Emmanuel Schanzer, Frances Spalding,
Jeff Vinocur, and David Walker.  

% Local Variables:
% TeX-master: "main-screen"
% End:
