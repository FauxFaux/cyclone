\section{Namespaces}

% The second reason that Cyclone rejects the C version is that it omits
% the \texttt{using} statement.  The \texttt{using} statement is
% necessary because Cyclone wraps all of its standard library
% definitions in C++ \emph{namespaces}.  A namespace is essentially a
% named scope: it associates a name with a set of declarations.  For
% example, the file \texttt{<stdio.h>} looks, in part, like this:
% \begin{alltt}
%     namespace Stdio \lb
%       {\it \ldots\ declarations \ldots}
%       extern int fflush(FILE *);
%     \rb
% \end{alltt}
% All of the declarations found in C's version of \texttt{<stdio.h>},
% including the one for the \texttt{fflush} function shown here, are
% contained within the namespace \texttt{Stdio}.  Once you've included
% the \texttt{<stdio.h>} file, you can refer to the variables in the
% namespace by prefixing them with \texttt{Stdio}, for example,
% \texttt{Stdio::fflush}.  If you don't want to use the \texttt{::}
% notation, you can make all of the variables within \texttt{Stdio}
% available with a \texttt{using} statement.  So, in our \texttt{hello}
% program, the statement \texttt{using Stdio;} allowed us to refer to
% the \texttt{printf} function as just \texttt{printf}, rather than
% \texttt{Stdio::printf}.
% 
% Namespaces allow two programmers to choose names without worrying that
% they will clash with each other.  For example, if you like you can
% define your own version of the \texttt{printf} function, and use your
% \texttt{printf} and \texttt{Stdio::printf} in the same program.  When
% namespaces are used properly, they isolate one programmer's internal
% variables from another's, enhancing safety.

As in C++, namespaces are used to avoid name clashes in code.  For
example:
\begin{verbatim}
  namespace Foo { 
    int x = 0; 
    int f() { return x; }  
  }
\end{verbatim}
declares an integer named \texttt{Foo::x} and a function named
\texttt{Foo::f}.  Note that within the namespace, you don't need to use
the qualified name.  For instance, \texttt{Foo::f} refers to
\texttt{Foo::x} as simply \texttt{x}.  We could also simply write
``\texttt{namespace Foo;}'' (note the trailing semi-colon) and leave out
the enclosing braces.  Every declaration (variables, functions, types,
typedefs) following this namespace declaration would be placed in the
Foo namespace.

As noted before, you can refer to elements of a namespace using the
``\texttt{::}'' notation.  Alternatively, you can open up a namespace
with a ``\texttt{using}'' declaration.  For example, we could follow the
above code with:
\begin{verbatim}
  namespace Bar {  
    using Foo { 
      int g() { return f(); } 
    } 
    int h() { return Foo::f(); } 
  }
\end{verbatim}

Here, we opened the \texttt{Foo} namespace within the definition of
\texttt{Bar::g}.  One can also write ``\texttt{using Foo;}'' to open a
namespace for the remaining definitions in the current block.

Namespaces can nest as in C++.  

Currently, namespaces are only supported at the top-level and you
can't declare a qualified variable directly.  Rather, you have to
write a namespace declaration to encapsulate it.  For example, you
cannot write ``\texttt{int Foo::x = 3;}.''

The following subtle issues and \textbf{implementation bugs} may leave
you scratching your head:
\begin{itemize}
\item The current implementation translates qualified Cyclone
  variables to C identifiers very naively: each \texttt{::} is
  translated to \texttt{_} (underscore).  This translation is wrong
  because it can introduce clashes that are not clashes in Cyclone,
  such as in the following:
\begin{verbatim}
  namespace Foo { int x = 7; }
  int Foo_x = 9;
\end{verbatim}
  
  So avoid prefixing your identifiers with namespaces in your program.
  We intend to fix this bug in a future release.
  
\item Because \texttt{\#include} is defined as textual substitution, the
  following are usually very bad ideas: Having ``\texttt{namespace Foo;}''
  or ``\texttt{using Foo;}'' at the top level of a header file.
  After all, you will be changing the identifiers produced or the
  identifiers available in every file that includes the header file.
  Having \texttt{\#include} directives within the scope of namespace
  declarations.  After all, you are changing the names of the
  identifiers in the header file by (further) qualifying them.
  Unfortunately, the current system uses the C pre-processor before
  looking at the code, so it cannot warn you of these probable errors.
  
  In short, you are advised to not use the ``semicolon syntax'' in
  header files and you are advised to put all \texttt{\#include}
  directives at the top of files, before any namespace or using
  declarations.
  
\item The translation of identifiers declared \texttt{extern "C"} is
  different.  Given
\begin{verbatim}
  namespace Foo { extern "C" int x; }
\end{verbatim}
  the Cyclone code refers to the global variable as \texttt{Foo::x}, but
  the translation to C will convert all uses to just \texttt{x}.  The
  following code will therefore get compiled incorrectly (\texttt{f}
  will return 4):
\begin{verbatim}
  namespace Foo { extern "C" int x; }
  int f() {
    int x = 2;
    return x + Foo::x;
  }
\end{verbatim}
\end{itemize}

% Local Variables:
% TeX-master: "main-screen"
% End:
