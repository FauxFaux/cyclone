\section{Libraries}

\ifscreen
\definecolor{cycdochighlight}{rgb}{.75,.81,1}
\else
\definecolor{cycdochighlight}{rgb}{1,1,1}
\fi

\subsection{C Libraries}
Cyclone provides partial support for the following C library headers:

\begin{tabular}{lll}\ttfamily
<aio.h> &
<arpa/inet.h> &
<assert.h> \\
<complex.h> &
<cpio.h> &
<ctype.h> \\
<dirent.h> &
<dlfcn.h> &
<errno.h> \\
<fcntl.h> &
<fenv.h> &
<float.h> \\
<fmtmsg.h> &
<fnmatch.h> &
<ftw.h> \\
<getopt.h> &
<glob.h> &
<grp.h> \\
<inttypes.h> &
<iso646.h> &
<langinfo.h> \\
<libgen.h> &
<limits.h> &
<locale.h> \\
<math.h> &
<monetary.h> &
<mqueue.h> \\
<ndbm.h> &
<net/if.h> &
<netdb.h> \\
<netinet/in.h> &
<netinet/tcp.h> &
<nl_types.h> \\
<poll.h> &
<pthread.h> &
<pwd.h> \\
<regexp.h> &
<sched.h> &
<search.h> \\
<semaphore.h> &
<setjmp.h> &
<signal.h> \\
<spawn.h> &
<stdarg.h> &
<stdbool.h> \\
<stddef.h> &
<stdint.h> &
<stdio.h> \\
<stdlib.h> &
<string.h> &
<strings.h> \\
<stropts.h> &
<sys/dir.h> &
<sys/file.h> \\
<sys/ioctl.h> &
<sys/ipc.h> &
<sys/mman.h> \\
<sys/msg.h> &
<sys/resource.h> &
<sys/select.h> \\
<sys/sem.h> &
<sys/shm.h> &
<sys/socket.h> \\
<sys/stat.h> &
<sys/statvfs.h> &
<sys/syslog.h> \\
<sys/time.h> &
<sys/timeb.h> &
<sys/times.h> \\
<sys/types.h> &
<sys/uio.h> &
<sys/un.h> \\
<sys/utsname.h> &
<sys/wait.h> &
<tar.h> \\
<termios.h> &
<tgmath.h> &
<time.h> \\
<trace.h> &
<ucontext.h> &
<ulimit.h> \\
<unistd.h> &
<utime.h> &
<utmpx.h> \\
<wchar.h> &
<wctype.h> &
<wordexp.h>
\end{tabular}




\input{gen-libdoc}

%HEVEA \newcommand{\parbox}[3][x]{#3}
\newcommand{\code}[1]{\texttt{#1}}
\newcommand{\var}[1]{\texttt{#1}}
\newcommand{\vvar}[1]{{\tt\textbf{\textit{#1}}}}
%HEVEA \begin{latexonly}
\ifscreen
\newcommand{\defunlabel}[1]{%
  \parbox[b]{\labelwidth}{\makebox[0pt][l]{\colorbox{lightblue}{#1}}}}
\else
\newcommand{\defunlabel}[1]{%
  \parbox[b]{\labelwidth}{\makebox[0pt][l]{#1}}}
\fi
\newenvironment{defun}[2]{%
  \begin{list}{}{}%
    \renewcommand{\makelabel}{\defunlabel}%
    \index{#1@\texttt{#1}}%
  \item[{\tt\textbf{#1}#2}]\mbox{}\\}{\end{list}}
\newenvironment{defun2}[3]{%
  \begin{list}{}{}%
    \renewcommand{\makelabel}{\defunlabel}%
    \index{#2@\texttt{#2()}}%
  \item[{\tt {#1}~\textbf{#2}{#3}}]\mbox{}\\}{\end{list}}
\newenvironment{deexn}[2]{%
  \begin{list}{}{}%
    \renewcommand{\makelabel}{\defunlabel}%
    \index{#1@\texttt{#1}}%
  \item[{\tt xtunion exn~\lb~extern~\textbf{#1}{#2}~\rb;}]\mbox{}\\}{\end{list}}
%HEVEA \end{latexonly}
%HEVEA \newcommand{\defunlabel}[1]{%
%HEVEA   \parbox[b]{\labelwidth}{%
%HEVEA \begin{rawhtml}<table><tr><td bgcolor="c0d0ff">\end{rawhtml}%
%HEVEA #1%
%HEVEA \begin{rawhtml}</td></tr></table>\end{rawhtml}%
%HEVEA }}
%HEVEA\newenvironment{defun}[2]{%
%HEVEA  \begin{list}{}{}%
%HEVEA    \renewcommand{\makelabel}{\defunlabel}%
%HEVEA    \index{#1@\texttt{#1}}%
%HEVEA  \item[{\tt\textbf{#1}#2}]}{\end{list}}
%HEVEA\newenvironment{defun2}[3]{%
%HEVEA  \begin{list}{}{}%
%HEVEA    \renewcommand{\makelabel}{\defunlabel}%
%HEVEA    \index{#2@\texttt{#2()}}%
%HEVEA  \item[{\tt {#1}~\textbf{#2}{#3}}]}{\end{list}}
%HEVEA\newenvironment{deexn}[2]{%
%HEVEA  \begin{list}{}{}%
%HEVEA    \renewcommand{\makelabel}{\defunlabel}%
%HEVEA    \index{#1@\texttt{#1}}%
%HEVEA  \item[{\tt xtunion exn~\lb~extern~\textbf{#1}{#2}~\rb;}]}{\end{list}}

\newcommand{\tindex}[1]{}
\newcommand{\exindex}[1]{}

% \subsection{Bitvec}

% Namespace \code{Bitvec} implements bit vectors, which can be used to
% represent sets with a fixed number of elements.

% \subsubsection*{Header}
% \begin{verbatim}
% #include <bitvec.h>
% using Bitvec;
% \end{verbatim}

% \subsubsection*{Types}
% \begin{verbatim}
% typedef int BITVEC[?];
% \end{verbatim}

% \subsubsection*{Creating bit vectors}

% \subsubsection*{Functions}
% \begin{verbatim}
% BITVEC new_empty(int);
% BITVEC new_full(int);
% BITVEC new_copy(BITVEC);
% BITVEC from_list<`a,`b>(Dict::Dict<`a,`b> d, int f(`b),int sz, List::list<`a>);
% List::list<int> to_sorted_list(BITVEC bvec, int sz);
% \end{verbatim}

% \subsubsection*{Use}

% \begin{defun}{new_empty}{(n)}
% Returns a bit vector containing \var{n} bits, all set to 0.
% \end{defun}

% \begin{defun}{new_full}{(n)}
% Returns a bit vector containing \var{n} bits, all set to 1.
% \end{defun}

% \begin{defun}{new_copy}{(v)}
% Returns a copy of bit vector \var{v}.
% \end{defun}

% \subsubsection*{Accessing and updating bits}

% \subsubsection*{Functions}
% \begin{verbatim}
% bool get(BITVEC, int);
% void set(BITVEC, int);
% void clear(BITVEC, int);
% bool get_and_set(BITVEC, int);
% void clear_all(BITVEC);
% void set_all(BITVEC);
% bool all_set(BITVEC bvec, int sz);
% \end{verbatim}

% \subsubsection*{Use}

% \begin{defun}{get}{(v,n)}
% Returns the \var{n}th bit of vector \var{v}.
% \end{defun}

% \begin{defun}{set}{(v,n)}
% Sets the \var{n}th bit of vector \var{v} to 1.
% \end{defun}

% \begin{defun}{clear}{(v,n)}
% Sets the \var{n}th bit of vector \var{v} to 0.
% \end{defun}

% \begin{defun}{get_and_set}{(v,n)}
% Sets the \var{n}th bit of vector \var{v} to 1, and returns the value
% that the \var{n}th bit had before it was set to 1.
% \end{defun}

% \begin{defun}{set_all}{(v)}
% Sets every bit of \var{v} to 1.
% \end{defun}

% \begin{defun}{clear_all}{(v)}
% Sets every bit of \var{v} to 0.
% \end{defun}

% \begin{defun}{all_set}{(v,n)}
% Does ?????  FIX
% \end{defun}

% \subsubsection*{Combining and comparing bit vectors}

% \subsubsection*{Functions}
% \begin{verbatim}
% void union_two(BITVEC dest, BITVEC src1, BITVEC src2);
% void intersect_two(BITVEC dest, BITVEC src1, BITVEC src2);
% void diff_two(BITVEC dest, BITVEC src1, BITVEC src2);
% bool compare_two(BITVEC src1, BITVEC src2);
% \end{verbatim}

% \subsubsection*{Use}

% \begin{defun}{union_two}{(dest,src1,src2)}
% Updates \var{dest} with the bitwise-or of \var{src1} and \var{src2}.
% There is no error if \var{dest} is shorter than \var{src1} or
% \var{src2} (extra bits are ignored).  If \var{src1} or \var{src2}
% is shorter than \var{dest}, \code{Null_Exception} is thrown;
% \var{dest} may be partially updated.
% \end{defun}

% \begin{defun}{intersect_two}{(dest,src1,src2)}
% Updates \var{dest} with the bitwise-and of \var{src1} and \var{src2}.
% Its error conditions are the same as \code{union_two}.
% \end{defun}

% \begin{defun}{diff_two}{(dest,src1,src2)}
% Updates \var{dest} with the bitwise-difference of \var{src1} and
% \var{src2}.  Its error conditions are the same as \code{union_two}.
% \end{defun}

% \begin{defun}{compare_two}{(src1,src2)}
% Returns 1 if every bit of
% \var{src1} is the same as the corresponding bit of \var{src2}, and 0
% otherwise.  If \var{src2} is longer than \var{src1}, its extra bits are
% ignored.  If \var{src2} is shorter than \var{src1},
% \code{Null_Exception} is thrown.
% \end{defun}

% \subsubsection*{Printing bit vectors}

% \subsubsection*{Functions}
% \begin{verbatim}
% void print_bvec(BITVEC bvec);
% \end{verbatim}

% \subsubsection*{Use}

% \begin{defun}{print_bvec}{(v)}
% FIX: PUT SOMETHING HERE
% \end{defun}

% \subsection{Ctype}

% Namespace \code{Ctype} implements the character library of C\@.

% \subsubsection*{Header}
% \begin{verbatim}
% #include <ctype.h>
% using Ctype;
% \end{verbatim}

% \subsubsection*{Functions}
% \begin{verbatim}
% int isalnum(int c);
% int isalpha(int c);
% int iscntrl(int c);
% int isdigit(int c);
% int isgraph(int c);
% int islower(int c);
% int isprint(int c);
% int ispunct(int c);
% int isspace(int c);
% int isupper(int c);
% int isxdigit(int c);
% int tolower(int c);
% int toupper(int c);
% int isascii(int c);
% int toascii(int c);
% \end{verbatim}

% \subsubsection*{Use}

% \begin{defun}{isalnum}{(c)}
% Returns 1 if \code{isalpha(c)} or \code{isnum(c)} returns 1,
% and returns 0 otherwise.
% \end{defun}

% \begin{defun}{isalpha}{(c)}
% Returns 1 if \code{islower(c)} or \code{isupper(c)} returns 1,
% and returns 0 otherwise.
% \end{defun}

% \begin{defun}{iscntrl}{(c)}
% Returns 1 if \var{c} is a control character,
% and returns 0 otherwise.
% \end{defun}

% \begin{defun}{isdigit}{(c)}
% Returns 1 if \var{c} is a decimal digit (\code{'0'}--\code{'9'}), and
% returns 0 otherwise.
% \end{defun}

% \begin{defun}{isgraph}{(c)}
% Returns 1 if \var{c} is a non-space printing character, and returns 0
% otherwise.
% \end{defun}

% \begin{defun}{islower}{(c)}
% Returns 1 if \var{c} is a lowercase letter (\code{'a'}--\code{'z'}),
% and returns 0 otherwise.
% \end{defun}

% \begin{defun}{isprint}{(c)}
% Returns 1 if \var{c} is a printing character (including space), and
% returns 0 otherwise.
% \end{defun}

% \begin{defun}{ispunct}{(c)}
% Returns 1 if \var{c} is a non-whitespace, non-digit, non-alpha
% printing character, and returns 0 otherwise.
% \end{defun}

% \begin{defun}{isspace}{(c)}
% Returns 1 if \var{c} is a whitespace character (\code{' '},
% \verb|'\f'|, \verb|'\n'|, \verb|'\r'|, \verb|'\t'|, or \verb|'\v'|),
% and returns 0 otherwise.
% \end{defun}

% \begin{defun}{isupper}{(\var{c})}
% Returns 1 if \var{c} is an uppercase letter
% (\code{'A'}--\code{'Z'}), and returns 0 otherwise.
% \end{defun}

% \begin{defun}{isxdigit}{(c)}
% Returns 1 if \var{c} is a hexidecimal digit (\code{'0'}--\code{'9'} or
% \code{'A'}--\code{'F'} or \code{'a'}--\code{'f'}), and returns 0
% otherwise.
% \end{defun}

% \begin{defun}{tolower}{(c)}
% If \var{c} is an uppercase letter (\code{'A'}--\code{'Z'}), then
% \code{tolower(\var{c})} returns its lowercase (\code{'a'}--\code{'z'})
% equivalent; otherwise it returns \var{c}.
% \end{defun}

% \begin{defun}{toupper}{(c)}
% If \var{c} is a lowercase letter (\code{'a'}--\code{'z'}), then
% \code{toupper(\var{c})} returns its uppercase
% (\code{'A'}--\code{'Z'}) equivalent; otherwise it returns \var{c}.
% \end{defun}

% \subsection{Stdio}

% \begin{verbatim}
% #include <stdio.h>
% using Stdio;
% \end{verbatim}

% The namespace \texttt{Stdio} provides input and output primitives.


% \subsubsection*{Formatted output}

% Cyclone provides safe versions of \code{printf} and \code{fprintf} as
% built-in primitives.  It also provides a built-in function,
% \code{aprintf}, that allocates a string, prints to it, and returns it.
% If the format string argument of these functions is given as a string
% constant, the types of the remaining arguments are checked at compile
% time; if the format string is not given as a string constant, the
% types of the remaining arguments are checked at run time, using
% Cyclone's safe vararg feature.  This prevents format string attacks in
% Cyclone.

% The permissible types for the varargs of \texttt{printf},
% \texttt{fprintf}, and \texttt{aprintf} are given by the type
% \texttt{parg_t}; if you need to know the details of \texttt{parg_t},
% consult \texttt{stdio.h}.

% \begin{defun2}{int}{fprintf}{(FILE @\var{f},const char {?}`r \var{fmt}, ...`r1 inject parg_t<`r2>);}
%   Prints to the stream pointed to by \var{f} according to the format
%   string \var{fmt} and the remaining arguments.  It returns the number
%   of characters printed, or a negative value if an output error or
%   encoding error occurred.
% \end{defun2}
% \begin{defun2}{int}{printf}{(const char {?}`r \var{fmt}, ...`r1 inject parg_t<`r2>);}
%   Prints to the standard output according to the format string \var{fmt}
%   and the remaining arguments.  It returns the number of characters
%   printed, or a negative value if an output error or encoding error
%   occurred.
% \end{defun2}
% \begin{defun2}{int}{sprintf}{(char {?}`r1 \var{s}, const char {?}`r2 \var{fmt}, ...`r3 inject parg_t<`r4>);}
%   Writes into string \var{s} according to format string \var{fmt} and
%   the remaining arguments.  It returns the number of characters
%   printed, or a negative value if an output error or encoding error
%   occurred.
% \end{defun2}
% \begin{defun2}{char ?}{aprintf}{(const char {?}`r2 \var{fmt}, ...`r3 inject parg_t<`r4>);}
%   Allocates and returns a string whose contents are given by printing
%   according to the format string \var{fmt} and the remaining arguments.
%   \texttt{FIX: is an error possible?  What do we do then?}
% \end{defun2}
% \begin{defun2}{char {?}`r1}{raprintf}{(region_t<`r1>, const char {?}`r2 fmt, ...`r3 inject parg_t<`r4> ap);}
%   Allocates and returns a string whose contents are given by printing
%   according to the format string \var{fmt} and the remaining arguments.
%   The string is allocated in region \code{`r1}.
% \end{defun2}

% \begin{defun2}{int}{vfprintf}{(FILE @,const char {?}`r fmt, parg_t<`r2> ? `r1 ap);}
%   A version of \code{fprintf} suitable for calling from a user's
%   vararg function.
% \end{defun2}
% \begin{defun2}{int}{vprintf}{(const char {?}`r fmt, parg_t<`r2> ? `r1);}
%   A version of \code{printf} suitable for calling from a user's
%   vararg function.
% \end{defun2}
% \begin{defun2}{int}{vsprintf}{(char {?}`r1 s, const char {?}`r2 fmt, parg_t<`r4> ? `r3);}
%   A version of \code{sprintf} suitable for calling from a user's
%   vararg function.
% \end{defun2}
% \begin{defun2}{char {?}`r1}{vraprintf}{(region_t<`r1> r1, const char {?}`r2 fmt, parg_t<`r4> ? `r3 ap);}
%   A version of \code{raprintf} suitable for calling from a user's
%   vararg function.
% \end{defun2}

% \subsubsection*{Output format strings}

% The format string consists of ordinary (non-\code{\%}) characters, which
% are copied to the output, and \emph{conversion specifications}, which
% are sequences of characters introduced by \code{\%}.  Each conversion
% specification `converts' zero or more arguments to the output, as
% described below.

% A conversion specification has the following parts, described, in order,
% by regular expressions:
% \begin{itemize}
% \item Zero or more \emph{flags}:
% [\code{-} \code{+} \emph{space} \code{\#} \code{0}]*
% \item An optional minimal \emph{width}:
% (\code{*} | [\code{0}-\code{9}]+)?
% \item An optional \emph{precision}:
% (\code{.} (\code{*} | [\code{0}-\code{9}]+)? )?
% \item An optional \emph{length modifier}:
% (\code{l} | \code{h} | \code{hh} )
% \item A \emph{conversion specifier}:
% [\code{d} \code{i} \code{o} \code{u} \code{x} \code{X} \code{f} \code{F} \code{e} \code{E} \code{g} \code{G} \code{a} \code{A} \code{c} \code{s} \code{n} \code{\%}]
% \end{itemize}

% The flag \code{-} causes left justification; right justification is the
% default.  The flag \code{+} is valid for numeric conversions only; it
% forces a \code{+} sign to be printed for positive arguments.  The flag
% \emph{space} (the space character) is valid for numeric conversions only;
% it causes an extra space to be printed in front of positive arguments.
% It is ignored if the \code{+} flag is also given.
% The flag \code{\#} is valid for \code{o}, \code{x}, \code{X}, \code{a},
% \code{A}, \code{e}, \code{E}, \code{f}, \code{F}, \code{g}, \code{G}
% only; it causes an ``alternative form'' to be printed.  For \code{o} it
% forces a leading zero to be printed; for \code{x} and \code{X}, a
% leading \code{0x} or \code{0X} is added to non-zero conversions; for the
% other conversions, a decimal point is always printed.

% The width of a conversion specification gives a minimum width for the
% result of the conversion; if the conversion naturally gives a shorter
% result, it is padded by spaces to the minimum width.  If the width has
% the form \code{*}, the minimum width is given by an integer argument.

% A precision is valid for numeric conversions only.  It gives the minimum
% number of digits to appear for the \code{d}, \code{i}, \code{o},
% \code{u}, \code{x}, and \code{X} conversions, the number of digits to
% appear after the decimal point in the \code{a}, \code{A}, \code{e},
% \code{E}, \code{f}, and \code{F} conversions, and the maximum number of
% significant digits for the \code{g} and \code{G} conversions.  If the
% precision has the form \code{.} \code{*}, it is given by an integer
% argument.

% The optional length modifier and conversion specifier give the type of
% the corresponding argument, as shown in the table below.  If no type
% appears, then the combination of conversion specifier and length
% modifier is not allowed.

% \begin{tt}
% \begin{tabular}{r|llll}
%   & none         & l             & h              & hh\\\hline
% d & int          & long          & short          & char \\
% i & int          & long          & short          & char \\
% o & unsigned int & unsigned long & unsigned short & unsigned char \\
% u & unsigned int & unsigned long & unsigned short & unsigned char \\
% x & unsigned int & unsigned long & unsigned short & unsigned char \\
% X & unsigned int & unsigned long & unsigned short & unsigned char \\
% f & double \\
% F & double \\
% e & double \\
% E & double \\
% g & double \\
% G & double \\
% a & double \\
% A & double \\
% c & int \\
% s & string \\
% n & int @        & unsigned long @ & short @      & char @ \\
% \% & \textrm{(no arg)}
% \end{tabular}
% \end{tt}

% No flags, width, precision, or length modifiers are allowed for the
% \code{s} and \code{\%} conversions.

% The \code{d} and \code{i} specifiers print their arguments as signed
% decimal integers.

% The \code{o} specifier prints its argument as an unsigned octal number.

% The \code{u} specifier prints its argument as an unsigned decimal number.

% The \code{x} and \code{X} specifiers print their arguments as unsigned
% hexidecimal integers.

% The \code{f} and \code{F} specifiers print their arguments in the form
% [-]ddd.ddd (no exponents).

% The \code{e} and \code{E} specifiers print their arguments using exponents.

% The \code{g} and \code{G} specifiers print their arguments using either
% the \code{f}/\code{F}
% style or the \code{e}/\code{E} style, depending on the precision and the
% argument.

% The \code{a} and \code{A} specifiers print their arguments as
% hexidecimal floating point numbers.

% The \code{c} specifier prints its argument as an unsigned character.

% The \code{s} specifier prints its argument as a zero terminated string.

% The \code{n} specifier does not print its argument; instead, the number
% of characters written up to the \code{n} specifier is stored where the
% argument points.

% The \code{\%} specifier prints a \code{\%} character.


% \subsubsection*{Formatted input}

% Cyclone provides safe versions of \code{scanf}, \code{fscanf}, and
% \code{sscanf}.  If the format string argument of these functions is
% given as a string constant, the types of the remaining arguments are
% checked at compile time; if the format string is not given as a string
% constant, the types of the remaining arguments are checked at run
% time, using Cyclone's safe vararg feature.

% The permissible types for the varargs of the input functions are given
% by the type \texttt{sarg_t}; if you need to know the details of
% \texttt{sarg_t}, consult \texttt{stdio.h}.

% \begin{defun2}{int}{scanf}{(const char {?}`r1 \var{fmt}, ...`r2 inject sarg_t<`r3,`r4>);}
%   Reads from the standard input according to the format string
%   \var{fmt}.  Results are stored in the remaining arguments.
%   \code{scanf} returns the number of arguments successfully read.  If
%   an error occurs before any arguments are converted, \code{EOF} is
%   returned.
% \end{defun2}
% \begin{defun2}{int}{fscanf}{(FILE @ \var{stream}, const char {?}`r1 \var{fmt}, ...`r2 inject sarg_t<`r3,`r4>);}
%   Reads from the \var{stream} according to the format string
%   \var{fmt}.  Results are stored in the remaining arguments.
%   \code{fscanf} returns the number of arguments successfully read.  If
%   an error occurs before any arguments are converted, \code{EOF} is
%   returned.
% \end{defun2}
% \begin{defun2}{int}{sscanf}{(const char {?}`r \var{src}, const char {?}`r1 \var{fmt}, ...`r2 inject sarg_t<`r3,`r4>);}
%   Reads from the string \var{src} according to the format string
%   \var{fmt}.  Results are stored in the remaining arguments.
%   \code{sscanf} returns the number of arguments successfully read.  If
%   an error occurs before any arguments are converted, \code{EOF} is
%   returned.
% \end{defun2}
% \begin{defun2}{int}{vfscanf}{(FILE @ stream, const char {?}`r1 \var{fmt}, sarg_t<`r3,`r4> {?}`r2);}
%   A version of \code{fscanf} suitable for calling from a user's
%   vararg function.
% \end{defun2}
% \begin{defun2}{int}{vsscanf}{(const char {?}`r \var{src}, const char {?}`r1 \var{fmt}, sarg_t<`r3,`r4> {?}`r2);}
%   A version of \code{sscanf} suitable for calling from a user's
%   vararg function.
% \end{defun2}

% \subsubsection*{Input format strings}

% The format string consists of white space characters (which match any
% (possible empty) sequence of white space characters), ordinary
% (non-\code{\%}) characters (which match themselves), and \emph{conversion
% specifications}, which are sequences of characters introduced by
% \code{\%}.  Each conversion specification `converts' a sequence of
% characters from the input stream, placing the result into storage
% pointed to by the corresponding argument of \code{scanf} or
% \code{fscanf}.

% A conversion specification has the following parts, described, in order,
% by regular expressions:
% \begin{itemize}
% \item An optional assignment suppression character, \code{*}
% \item An optional maximal \emph{width}:
% ([\code{0}-\code{9}]+)?
% \item An optional \emph{length modifier}:
% (\code{l} | \code{h} | \code{hh} )
% \item A \emph{conversion specifier}:
% [\code{d} \code{i} \code{o} \code{u} \code{x} \code{X} \code{f} \code{F} \code{e} \code{E} \code{g} \code{G} \code{a} \code{A} \code{n} \code{\%}]
% \end{itemize}

% If the assignment suppression character \code{*} is given, the
% characters corresponding to the conversion specifier are read, but no
% argument is assigned, and no argument is required for that specifier.

% The optional length modifier and conversion specifier give the type of
% the corresponding argument, as shown in the table below.  If no type
% appears, then the combination of conversion specifier and length
% modifier is not allowed.

% \begin{tt}
% \begin{tabular}{r|llll}
%   & none           & l               & h                &  hh\\\hline
% d & int @          & long @          & short @          & char @ \\
% i & int @          & long @          & short @          & char @ \\
% o & unsigned int @ & unsigned long @ & unsigned short @ & unsigned char @ \\
% u & unsigned int @ & unsigned long @ & unsigned short @ & unsigned char @ \\
% x & unsigned int @ & unsigned long @ & unsigned short @ & unsigned char @ \\
% X & unsigned int @ & unsigned long @ & unsigned short @ & unsigned char @ \\
% f & float @        & double @ \\
% F & float @        & double @ \\
% e & float @        & double @ \\
% E & float @        & double @ \\
% g & float @        & double @ \\
% G & float @        & double @ \\
% a & float @        & double @ \\
% A & float @        & double @ \\
% n & int @          & unsigned long @ & short @          & char @ \\
% \% & \texttt{(no arg)}
% \end{tabular}
% \end{tt}

% The \code{d} and \code{i} specifiers match optionally signed decimal
% integers.

% The \code{o} specifier matches an optionally signed octal integer.

% The \code{u} specifier matches an optionally signed decimal integer.

% The \code{x} and \code{X} specifiers match optionally signed hexidecimal
% integers.

% The \code{f}, \code{F}, \code{e}, \code{E}, \code{g}, \code{G},
% \code{a}, and \code{A} specifiers match floating point numbers.

% The \code{n} specifier does not match any input; instead, the number of
% characters read by the call of \code{fscanf} or \code{scanf} up to the
% \code{n} specifier is stored where the argument points.

% The \code{\%} specifier matches a \code{\%} character.  No assignment
% suppression, width, or length modifiers are allowed for the \code{\%}
% conversion.

% \subsection{String}

% \begin{verbatim}
%     #include <string.h>
%     using String;
% \end{verbatim}

% Namespace \code{String} implements a string library similar to the one
% provided by C\@.  In Cyclone, a string is a \code{char ?}.  Because
% strings are so pervasive, several common typedefs for strings are
% defined in namespace \texttt{Core}:
% \begin{alltt}
%     typedef const char ?\textbf{string_t};
%     typedef char ?\textbf{mstring_t};
%     typedef string_t @\textbf{stringptr_t};
%     typedef mstring_t @\textbf{mstringptr_t};
% \end{alltt}
% A \texttt{string_t} is an immutable string, while an
% \texttt{mstring_t} is a mutable string.  The pointer variants of these
% are useful to instantiate strings in polymorphic data types.  We use
% ``string'' to refer to any and all of these, depending on context.

% Most functions in \texttt{String} consider a zero (NUL) character as
% an end of string marker, so we may have \texttt{strlen(s)} <
% \texttt{s.size} for a string \var{s}.  Functions that do not follow
% this convention have names starting with 'z'.  Many functions accept a
% string and an offset; these functions may have names ending in 'o'.
% Functions whose arguments have range errors throw the
% \code{Core::InvalidArg} exception.

% \subsubsection*{String length}
% \begin{defun2}{size_t}{strlen}{(const char {?}`r \vvar{s});}
%   Returns the length of the string \vvar{s}.  It considers a NUL
%   character to mark the end of the string.
% \end{defun2}

% \subsubsection*{Comparing strings}
% The string comparison functions return an integer less than, equal to,
% or greater than 0 if their first argument is less than, equal to, or
% greater than their second argument, respectively.  The ordering used
% is the standard, lexicographic ordering.

% \begin{defun2}{int}{strcmp}{(const char {?}`r1 \vvar{s1}, const char {?}`r2 \vvar{s2});}
%   Compares \var{s1} to \var{s2}.
% \end{defun2}

% \begin{defun2}{int}{strptrcmp}{(const char {?}`r1@`r2 \vvar{s1}, const char {?}`r3@`r4 \vvar{s2});}
%   A string pointer version of \code{strcmp}.
% \end{defun2}

% \begin{defun2}{int}{strncmp}{(const char {?}`r1 \vvar{s1}, const char {?}`r2 \vvar{s2}, size_t \vvar{len});}
%   Compares at most \vvar{len} characters of \vvar{s1} to \vvar{s2}.  If
%   \vvar{len} is negative, \code{strncmp} returns 0.
% \end{defun2}

% \begin{defun2}{int}{strncasecmp}{(const char {?}`r1 \vvar{s1}, const char {?}`r2 \vvar{s2}, size_t \vvar{len});}
%   A case-insensitive version of \code{strncmp}.
% \end{defun2}

% \begin{defun2}{int}{zstrcmp}{(const char {?}`r1 \vvar{s1},const char {?}`r2 \vvar{s2});}
%   Compares \vvar{s1} to \vvar{s2}, and it assumes that NUL (zero)
%   characters in \vvar{s1} and \vvar{s2} are not end-of-string markers.
% \end{defun2}

% \begin{defun2}{int}{zstrptrcmp}{(const char {?}`r1@`r2 \vvar{s1}, const char {?}`r3@`r4 \vvar{s2});}
%   A string pointer version of \code{zstrcmp}.
% \end{defun2}

% \begin{defun2}{int}{zstrncmp}{(const char {?}`r1 \vvar{s1},const char {?}`r2 \vvar{s2}, size_t \vvar{n});}
%   Compares at most \vvar{n} characters of \vvar{s1} to \vvar{s2},
%   assuming that NUL characters do not mark the end of strings.  If
%   \vvar{n} is less than 0, \code{zstrncmp} returns 0.
% \end{defun2}

% \subsubsection*{Concatenating strings}

% \begin{verbatim}
% \end{verbatim}

% \begin{defun2}{char {?}`r1}{strcat}{(char {?}`r1 \vvar{dest},const char {?}`r2 \vvar{src});}
%   Concatenates \vvar{src} onto \vvar{dest} and returns \vvar{dest}.
%   If \vvar{dest} is not large enough, \code{strcat} throws
%   \code{Core::InvalidArg("String::strcat")}.
% \end{defun2}

% \begin{defun2}{char ?}{strconcat}{(const char {?}`r1 \vvar{s1},const char {?}`r2 \vvar{s2});}
%   Heap allocates and returns a new string whose contents are the
%   concatenation of \vvar{s1} and \vvar{s2}.
% \end{defun2}

% \begin{defun2}{char {?}`r}{rstrconcat(region_t<`r> \vvar{r},const char {?}`r1 \vvar{dest},const char {?}`r2 \vvar{src});}
%   A version of \code{strconcat} that allocates the resulting string in
%   the region given by \vvar{r}.
% \end{defun2}

% \begin{defun2}{char ?}{strconcat_l}{(glist_t<const char {?}`r1@`r2,`r> \vvar{l});}
%   Allocates and returns a new string whose contents are the
%   concatenation of the strings in the list \vvar{l}, from left to
%   right.
% \end{defun2}

% \begin{defun2}{char {?}`r}{rstrconcat_l}{(region_t<`r> \vvar{r},glist_t<const char {?}`r1@`r2,`r3> \vvar{l});}
%   A version of \code{strconcat_l} that allocates the resulting string
%   in the region given by \vvar{r}.
% \end{defun2}

% \begin{defun2}{char ?}{str_sepstr}{(glist_t<const char {?}`r1@`r2,`r> \vvar{l},const char {?}`r3 \vvar{s});}
%   Allocates and returns a new string whose contents are the
%   concatenation of the strings in the list \vvar{l}, with \vvar{s} used
%   as a separator between each two adjacent elements in \vvar{l}.
% \end{defun2}

% \begin{defun2}{char {?}`r}{rstr_sepstr}{(\begin{tabular}[t]{@{}l@{}}
%       region_t<`r> \vvar{r},\\
%       glist_t<const char {?}`r1@`r2,`r3> \vvar{l},\\
%       const char {?}`r4 \vvar{s});\end{tabular}}
%   A version of \code{str_sepstr} that allocates the resulting string
%   in the region given by \vvar{r}.
% \end{defun2}

% \subsubsection*{Copying strings and substrings}
% \subsubsection*{Functions}
% \begin{verbatim}
% string strcpy(string dest,string src);
% string strncpy(string,string,size_t);
% string zstrncpy(string,string,size_t);
% string expand(string s, size_t sz);
% string realloc_str(string str, size_t sz);
% string strdup(string src);
% string substring(string,int ofs, size_t n);
% \end{verbatim}

% \subsubsection*{Use}

% \begin{defun}{strcpy}{(dest,src)}
% Copies \var{src} into \var{dest} and returns \var{dest}.  If \var{dest}
% is not big enough to hold \var{src}, \code{strcpy} throws
% \code{Core::InvalidArg("String::strncpy")}.
% \end{defun}

% \begin{defun}{strncpy}{(dest,src,len)}
% Copies at most \var{len} characters of \var{src} into \var{dest} and
% returns \var{dest}.  If \var{dest} is not big enough to hold
% \var{src}, \code{strcpy} throws
% \code{Core::InvalidArg("String::strncpy")}.
% \end{defun}

% \begin{defun}{zstrncpy}{(dest,src,len)}
% A variant of \code{strncpy} that does not consider NUL characters to
% terminate strings.
% \end{defun}

% \begin{defun}{expand}{(s,n)}
% Allocates and returns a new string that has size \var{n} or
% \code{strlen(\var{s})}, whichever is greater.  The returned string has
% the same contents as \var{s} (considering NUL characters as string
% terminators).
% \end{defun}

% \begin{defun}{realloc_str}{(s,n)}
% Like expand except that the returned string is usually a bit bigger than
% \var{n} and \code{strlen(\var{s})}.
% \end{defun}

% \begin{defun}{strdup}{(s)}
% Allocates and returns a new string with the same contents as \var{s}
% (assuming that NUL characters terminate strings).
% \end{defun}

% \begin{defun}{substring}{(s,ofs,len)}
% Allocates and returns a new string whose contents are \var{len}
% characters of \var{s} starting at offset \var{ofs}.  If \var{ofs} or
% \var{len} are out of bounds then substrings throws
% \code{Core::InvalidArg("String::substring")}.
% \end{defun}

% \subsubsection*{Transforming strings}
% \subsubsection*{Functions}
% \begin{verbatim}
% string replace_suffix(string,string,string);
% \end{verbatim}

% \subsubsection*{Use}

% \begin{defun}{replace_suffix}{(s,before,after)}
% Allocates and returns a string whose characters are the characters of
% \var{s}, with its suffix before replaced by the string after.  If before
% is not a suffix of \var{s}, then \code{replace_suffix} throws
% \code{Core::InvalidArg("String::replace_suffix")}.
% \end{defun}

% \subsubsection*{Searching in strings}
% \subsubsection*{Functions}
% \begin{verbatim}
% int strchr(string s, int ofs, char c);
% int strrchr(string s, int ofs, char c);
% int strpbrk(string s, int ofs, string accept);
% int strspn(string s, int ofs, string accept);
% \end{verbatim}

% \subsubsection*{Use}

% \begin{defun}{strchr}{(s,ofs,c)}
% Returns the lowest index \code{\var{i} >= \var{ofs}} such that
% \code{\var{s}[\var{i}] == \var{c}}.  If \var{ofs} is out of range strchr
% throws \code{Core::InvalidArg("String::strchr")}.  If \var{c} does not
% appear in \var{s} starting at \var{ofs}, \code{strchr} returns -1.
% \end{defun}

% \begin{defun}{strrchr}{(s,ofs,c)}
% Returns the greatest index \code{\var{i} >= \var{ofs}} such that
% \code{\var{s}[\var{i}] == \var{c}}.  If \var{ofs} is out of range,
% \code{strrchr} throws \code{Core::InvalidArg("String::strrchr")}.  If
% \var{c} does not appear in \var{s} starting at \var{ofs}, \code{strrchr}
% returns -1.
% \end{defun}

% \begin{defun}{strpbrk}{(s,ofs,accept)}
% Returns the lowest index \code{\var{i} >= \var{ofs}} such that
% \code{\var{s}[\var{i}] == \var{c}}, where \var{c} is any character of
% accept.  If \var{ofs} is out of range, strpbrk throws
% \code{Core::InvalidArg("String::strpbrk")}.  If no character of accept
% appears in \var{s} starting at \var{ofs}, \code{strpbrk} returns -1.
% \end{defun}

% \begin{defun}{strspn}{(s,ofs,accept)}
% Returns the lowest index \code{\var{i} >= \var{ofs}} such that
% \code{\var{s}[\var{i}] == \var{c}}, where \var{c} is any character not
% in the string \var{accept}.  If \var{ofs} is out of range, \code{strspn}
% throws \code{Core::InvalidArg("String::strpbrk")}.  If no character of
% \var{accept} appears in \var{s} starting at \var{ofs}, \code{strspn}
% returns the length of \var{s} after \var{ofs}.
% \end{defun}


% \subsubsection*{String conversions}
% \subsubsection*{Functions}
% \begin{verbatim}
% list<Char> explode(string s);
% string implode(list<Char> c);
% \end{verbatim}

% \subsubsection*{Use}

% \begin{defun}{explode}{(s)}
% Returns a list of the characters in string \var{s}.
% \end{defun}

% \begin{defun}{implode}{(l)}
% Allocates and returns a string whose contents are the characters in list
% \var{l}, from left to right.
% \end{defun}

% Local Variables:
% TeX-master: "main-screen"
% End:
