\section{Porting C code to Cyclone}

%HEVEA \begin{latexonly}
\ifscreen
\newenvironment{porta}[2]{%
  \begin{list}{}{}%
  \item[\hypertarget{#1}{\colorbox{lightblue}{\textbf{#2}}}]}{\end{list}}
\else
\newenvironment{porta}[2]{%
  \begin{list}{}{}%
  \item[\hypertarget{#1}{\textbf{#2}}]}{\end{list}}
\fi
%HEVEA \end{latexonly}
%HEVEA \newenvironment{porta}[2]{%
%HEVEA   \begin{list}{}{}%
%HEVEA   \item[\hypertarget{#1}{%
%HEVEA \begin{rawhtml}<table><tr><td bgcolor="c0d0ff">\end{rawhtml}%
%HEVEA \textbf{#2}%
%HEVEA \begin{rawhtml}</td></tr></table>\end{rawhtml}%
%HEVEA }]}{\end{list}}

Though Cyclone resembles and shares a lot with C, porting is not
always straightforward.  Furthermore, it's rare that you actually port
an entire application to Cyclone.  You may decide to leave certain
libraries or modules in C and port the rest to Cyclone.  In this
Chapter, we want to share with you the tips and tricks that we have
developed for porting C code to Cyclone and interfacing Cyclone
code against legacy C code.

\subsection{Translating C to Cyclone}

To a first approximation, you can port a simple program
from C to Cyclone by following these steps which are
detailed below:
\begin{itemize}
\item \hyperlink{port:null}{Use NULL instead of 0.}
\item \hyperlink{port:pointers}{Change pointer types to fat pointer types where necessary.}  
\item \hyperlink{port:malloc}{Use comprehensions to heap-allocate arrays.}
\item \hyperlink{port:unions}{Use tunions for unions with pointers.}
\item \hyperlink{port:initialize}{Initialize variables.}  
\item \hyperlink{port:cases}{Put breaks or fallthrus in switch cases.} 
\item \hyperlink{port:temp}{Replace one temporary with multiple temporaries.}  
\item \hyperlink{port:rgns}{Connect argument and result pointers with the same region.}
\item \hyperlink{port:types}{Insert type information to direct the type-checker.}
\item \hyperlink{port:const}{Copy ``const'' code or values to make it non-const.}  
\item \hyperlink{port:free}{Get rid of calls to free, realloc, memset, memcpy, etc.}
\item \hyperlink{port:poly}{Use polymorphism or tunions to get rid of void*.}  
\item \hyperlink{port:arrays}{Be careful with \& and arrays.}
\item \hyperlink{port:vararg}{Rewrite the bodies of vararg functions.}
\item \hyperlink{port:setjmp}{Use exceptions instead of setjmp.}
\end{itemize}

Even when you follow these suggestions, you'll still need to test and
debug your code carefully.  By far, the most common run-time errors
you will get are uncaught exceptions for null-pointer dereference
or array out-of-bounds.  Under Linux, you should get a stack backtrace
when you have an uncaught exception which will help narrow down
where and why the exception occurred.  On other architectures, you
can use \texttt{gdb} to find the problem.  The most effective way
to do this is to set a breakpoint on the routines \texttt{\_throw\_null()}
and \texttt{\_throw\_arraybounds()} which are defined in the
runtime and used whenever a null-check or array-bounds-check fails.
Then you can use \texttt{gdb}'s backtrace facility to see where
the problem occurred.  Of course, you'll be debugging at the C
level, so you'll want to use the \texttt{-save-c} and \texttt{-g}
options when compiling your code.  

\begin{porta}
{port:null}{Use NULL instead of 0.}

Use \texttt{NULL} instead of 0 for null-pointers.  
\end{porta}

\begin{porta}
{port:pointers}{Change pointer types to fat pointer types where necessary.}  

Ideally, you should examine the code and use thin pointers (e.g., \texttt{int*}
or better \texttt{int@}) wherever possible as these require fewer
run-time checks and less storage.  However, recall that thin pointers
do not support pointer arithmetic.  In those situations, you'll need
to use fat pointers (e.g., \texttt{int?}).  A particularly simple strategy 
when porting C code is to just change all pointers to fat pointers.
The code is then more likely to compile, but will have greater overhead.
After changing to use all fat pointers, you may wish to profile or reexamine
your code and figure out where you can profitably use thin pointers.
\end{porta}

\begin{porta}{port:malloc}{Use comprehensions to heap-allocate arrays.}

Cyclone provides limited support for \texttt{malloc} and separated
initialization but this really only works for \texttt{struct}s or
tuples.  To heap- or region-allocate and initialize an array, use
\texttt{new} or \texttt{rnew} in conjunction with array comprehensions.  
For example, to copy a string \texttt{s}, one might write:
\begin{verbatim}
  char ?t = new {for i < s.size : s[i]};
\end{verbatim}


\end{porta}

\begin{porta}{port:unions}{Use tunions for unions with pointers.}

Cyclone only accepts unions that contain ``bits'' (i.e., ints; chars;
shorts; floats; doubles; or tuples, structs, unions, or arrays of bits.)
So if you have a C union with a pointer type in it, you'll have to
code around it.  One way is to simply use a tagged union (\texttt{tunion}).
Note that this adds a level of indirection and requires pattern
matching to ensure type-safety.  

\end{porta}

\begin{porta}{port:initialize}{Initialize variables.}  

Top-level variables must be initialized
in Cyclone, and in many situations, local variables must be initialized.
Sometimes, this will force you to change the type of the variable
so that you can construct an appropriate initial value.  For instance,
suppose you have the following declarations at top-level:
\begin{verbatim}
struct DICT; 
struct DICT @new_dict();
struct DICT @d;
void init() {
  d = new_dict();
}
\end{verbatim}
Here, we have an abstract type for dictionaries 
(\texttt{struct Dict}), a constructor
function (\texttt{new\_dict()}) which returns a pointer to a new
dictionary, and a top-level variable (\texttt{d}) which is meant
to hold a pointer to a dictionary.  The \texttt{init} function
ensures that \texttt{d} is initialized.  However, 
Cyclone would complain that 
\texttt{d} is not initialized because \texttt{init} may not be
called, or it may only be called after \texttt{d} is already used.  
Furthermore, the only way to initialize \texttt{d}
is to call the constructor, and such an expression is not a 
valid top-level initializer.  The solution is to declare \texttt{d} as
a ``possibly-null'' pointer to a dictionary and initialize it
with \texttt{NULL}:
\begin{verbatim}
struct DICT; 
struct DICT @new_dict();
struct DICT *d;
void init() {
  d = new_dict();
}
\end{verbatim}
Of course, now whenever you use \texttt{d}, either you or the compiler
will have to check that it is not \texttt{NULL}.
\end{porta}

\begin{porta}{port:cases}{Put breaks or fallthrus in switch cases.}  

Cyclone requires
that you either break, return, continue, throw an exception, or explicitly
fallthru in each case of a switch.  
\end{porta}

\begin{porta}{port:temp}{Replace one temporary with multiple temporaries.}

Consider the following code:
\begin{verbatim}
void foo(char ? x, char ? y) {
  char ? temp;
  temp = x;
  bar(temp);
  temp = y;
  bar(temp);
}
\end{verbatim}
When compiled, Cyclone generates an error message like this:
\begin{verbatim}
type mismatch: const unsigned char ?#0  != unsigned char ?#1 
\end{verbatim}
The problem is that Cyclone thinks that \texttt{x} and \texttt{y}
might point into different regions (which it named \texttt{\#0} and
\texttt{\#1} respectively), and the variable \texttt{temp} is assigned
both the value of \texttt{x} and the value of \texttt{y}.  Thus,
there is no single region that we can say \texttt{temp} points into.
The solution in this case is to use two different temporaries for
the two different purposes:
\begin{verbatim}
void foo(char ? x, char ? y) {
  char ? temp1;
  char ? temp2;
  temp1 = x;
  bar(temp1);
  temp2 = y;
  bar(temp2);
}
\end{verbatim}
Now Cyclone can figure out that \texttt{temp1} is a pointer into
the region \texttt{\#0} whereas \texttt{temp2} is a pointer into
region \texttt{\#1}.  
\end{porta}

\begin{porta}{port:rgns}{Connect argument and result pointers with the same region.}

Remember that Cyclone assumes that pointer inputs to a function might
point into distinct regions, and that output pointers, by default point
into the heap.  Obviously, this won't always be the case.  Consider
the following code:
\begin{verbatim}
int @foo(int @x, int @y, int b) {
  if (b)
    return x;
  else
    return y;
}
\end{verbatim}
Cyclone complains when we compile this code:
\begin{verbatim}
returns value of type int @#0  but requires int @
        #0 and `H failed to unify. 
returns value of type int @#1  but requires int @
        #1 and `H failed to unify. 
\end{verbatim}
reflecting the fact that neither \texttt{x} nor \texttt{y} is a pointer
into the heap.  You can fix this problem by putting in explicit regions
to connect the arguments and the result.  For instance, we might write:
\begin{verbatim}
int @`r foo(int @`r x, int @`r y, int b) {
  if (b)
    return x;
  else
    return y;
}
\end{verbatim}
and then the code will compile.  Of course, any caller to this function
must now ensure that the arguments are in the same region.  
\end{porta}

\begin{porta}{port:types}{Insert type information to direct the type-checker.}

Cyclone is usually good about inferring types.  But sometimes, it
has too many options and picks the wrong type.  A good example is
the following:
\begin{verbatim}
void foo(int b) {
  printf("b is %s", b ? "true" : "false");
} 
\end{verbatim}
When compiled, Cyclone warns:
\begin{verbatim}
(2:39-2:40): implicit cast to shorter array
\end{verbatim}
The problem is that the string \texttt{"true"} is assigned the
type \texttt{const char ?\{5\}} whereas the string
\texttt{"false"} is assigned the type \texttt{const char ?\{6\}}.
(Remember that string constants have an implicit 0 at the end.)
The type-checker needs to find a single type for both since
we don't know whether \texttt{b} will come out true or false
and conditional expressions require the same type for either
case.  There are at least two ways that the types of the strings can be
promoted to a unifying type.  One way is to promote both
to \texttt{char?} which would be ideal.  Unfortunately, Cyclone
has chosen another way, and promoted the longer string
(\texttt{"false"}) to a shorter string type, namely
\texttt{const char ?\{5\}}.  This makes the two types the
same, but is not at all what we want, for when the procedure
is called with false, the routine will print
\begin{verbatim}
b is fals
\end{verbatim}
Fortunately, the warning indicates that there might be a problem.
The solution in this case is to explicitly cast at least one of the two
values to \texttt{const char ?}:
\begin{verbatim}
void foo(int b) {
  printf("b is %s", b ? ((const char ?)"true") : "false");
} 
\end{verbatim}
Alternatively, you can declare a temp with the right type and use
it:
\begin{verbatim}
void foo(int b) {
  const char ? t = b ? "true" : "false"
  printf("b is %s", t);
} 
\end{verbatim}
The point is that by giving Cyclone more type information, you can
get it to do the right sorts of promotions.  
\end{porta}

\begin{porta}{port:const}{Copy ``const'' code or values to make it non-const.}  

Cyclone takes \texttt{const} seriously.  C does not.  Occasionally,
this will bite you, but more often than not, it will save you from
a core dump.  For instance, the following code will seg fault on
most machines:
\begin{verbatim}
void foo() {
  char ?x = "howdy"
  x[0] = 'a';
}
\end{verbatim}
The problem is that the string \texttt{"howdy"} will be placed in
the read-only text segment, and thus trying to write to it will
cause a fault.  Fortunately, Cyclone complains that you're trying
to initialize a non-const variable with a const value so this
problem doesn't occur in Cyclone.  If you really want to initialize
\texttt{x} with this value, then you'll need to copy the string,
say using the \texttt{dup} function from the string library:
\begin{verbatim}
void foo() {
  char ?x = dup("howdy");
  x[0] = 'a';
}
\end{verbatim}
Now consider the following
call to the \texttt{strtoul} code in the standard library:
\begin{verbatim}
extern unsigned long strtoul(const char ?`r n, 
                             const char ?`r*`r2 endptr,
                             int base);

unsigned long foo() {
  char ?x = dup("howdy");
  char ?*e = NULL;
  return strtoul(x,e,0);
}
\end{verbatim}
Here, the problem is that we're passing non-const values to the
library function, even though it demands const values.  Usually,
that's okay, as \texttt{const char ?} is a super-type of
\texttt{char ?}.  But in this case, we're passing as the
\texttt{endptr} a pointer to a \texttt{char ?}, and it
is not the case that \texttt{const char ?*} is a super-type
of \texttt{char ?*}.  In this case, you have two options:
Either make \texttt{x} and \texttt{e} const, or copy the
code for \texttt{strtoul} and make a version that doesn't
have const in the prototype.  
\end{porta}

\begin{porta}{port:free}{Get rid of calls to free, realloc, memset, memcpy, etc.}

There are many standard functions that Cyclone can't support 
and still maintain type-safety.  An obvious one is \texttt{free()}
which releases memory.  Let the garbage collector free the object
for you, or use region-allocation if you're scared of the collector.
Other operations, such as \texttt{memset} and \texttt{memcpy} 
are also not supported
by Cyclone. You'll need to write code to manually copy one data
structure to another.  Fortunately, this isn't so bad since Cyclone
supports structure assignment.  
\end{porta}

\begin{porta}{port:poly}{Use polymorphism or tunions to get rid of void*.}  
Often you'll find C code that uses \texttt{void*} to simulate
polymorphism.  A typical example is something like swap:
\begin{verbatim}
void swap(void **x, void **y) {
  void *t = x;
  x = y;
  y = t;
}
\end{verbatim}
In Cyclone, this code should type-check but you won't be able
to use it in many cases.  The reason is that while \texttt{void*}
is a super-type of just about any pointer type, it's not the
case that \texttt{void**} is a super-type of a pointer to a
pointer type.  In this case, the solution is to use Cyclone's
polymorphism:
\begin{verbatim}
void swap(`a @x, `a @y) {
  `a t = x;
  x = y;
  y = t;
}
\end{verbatim}
Now the code can (safely) be called with any two (compatible)
pointer types.  This trick works well as long as you only need
to ``cast up'' from a fixed type to an abstract one.  It doesn't
work when you need to ``cast down'' again.  For example, consider
the following:
\begin{verbatim}
int foo(int x, void *y) {
  if (x)
   return *((int *)y);
  else {
    printf("%s\n",(char *)y);
    return -1;
  }
}
\end{verbatim}
The coder intends for \texttt{y} to either be an int pointer or
a string, depending upon the value of \texttt{x}.  If \texttt{x}
is true, then \texttt{y} is supposed to be an int pointer, and
otherwise, it's supposed to be a string.  In either case, you have
to put in a cast from \texttt{void*} to the appropriate type,
and obviously, there's nothing preventing someone from passing
in bogus cominations of \texttt{x} and \texttt{y}.  The solution
in Cylcone is to use a tagged union to represent the dependency
and get rid of the variable \texttt{x}:
\begin{verbatim}
tunion IntOrString { Int(int), String(char ?) };
typedef tunion IntOrString i_or_s;

int foo(i_or_s y) {
  switch (y) {
  case Int(i):  return i;
  case String(s):  
    printf("%s\n",s);
    return -1;
  }
}
\end{verbatim}
\end{porta}

\begin{porta}{port:arrays}{Be careful with \& and arrays.}

There's currently a bug in Cyclone that involves a very subtle problem
with arrays.  The issue is that in C, the expressions
``\texttt{\&A[i]}'', ``\texttt{A+i}'' and ``\texttt{i+A}'' are
all equivalent.  In Cyclone, this is currently not the case because
of array bounds checking.  Consider the following code:
\begin{verbatim}
  int Arr[100] = {for i < 100 : 0};
  int ?A = Arr;
  int ?B = A+3
  int ?C = &A[3];
  B[77] = C[77];
\end{verbatim}
The code looks perfectly reasonable but it will cause an out-of-bounds
exception.  The problem is that \texttt{A} will be considered an
array of size 100, \texttt{B} will be considered an array of size
97, but \texttt{C} will be considered an array of size 1!  In
particular, the expression ``\texttt{\&A[3]}'' has type
\texttt{int@} $=$ \texttt{int@\{1\}} which is a sub-type of
\texttt{int?}, so the initialization of \texttt{C} is type-correct.
However, while passing through the type \texttt{int@\{1\}} the
size of the array is effectively truncated to one.  Consequently,
the expression ``\texttt{C[77]}'' will cause an out-of-bounds
exception.  

\end{porta}

\begin{porta}{port:vararg}{Rewrite the bodies of vararg functions.}

See the section on varargs for more details.  
\end{porta}

\begin{porta}{port:setjmp}{Use exceptions instead of setjmp.}

Many uses of \texttt{setjmp}/\texttt{longjmp} can be replaced
with a \texttt{try}-block and a \texttt{throw}.  Of course,
you can't do this for things like a user-level threads package,
but rather, only for those situations where you're trying
to ``pop-out'' of a deeply nested set of function calls.
\end{porta}

\subsection{Interfacing to C}

When porting any large code from C to Cyclone, or even when writing
a Cyclone program from scratch, you'll want to be able to access
legacy libraries.  To do so, you must understand how Cyclone
represents data structures, how it compiles certain features,
and how to write wrappers to make up for representation mismatches.

Sometimes, interfacing to C code is as simple as writing
an appropriate interface.  For instance, if you want to
call the \texttt{acos} function which is defined in the C
Math library, you can simply write the following:
\begin{verbatim}
extern "C" double acos(double);
\end{verbatim}
The \texttt{extern "C"} scope declares that the function is
defined externally by C code.  As such, it's name is not
prefixed with any namespace information by the compiler.
Note that you can still embed the function within a Cyclone
namespace, it's just that the namespace is ignored by the
time you get down to C code.  

If you have a whole group of functions then you can wrap them with
a single \texttt{extern "C" \{ ... \}}, as in:
\begin{verbatim}
extern "C" {
  double acos(double);
  float  acosf(float);
  double acosh(double);
  float  acoshf(float);
  double asin(double);
}
\end{verbatim}

The extern C approach works well enough that it covers many
of the cases that you'll encounter.  However, the situation is
not so easy or straightforward when you start to take advantage
of Cyclone's features.  As a simple example, suppose you want
to call a C function \texttt{int\_to\_string} that takes in
an integer and returns a string representation of that integer.
The C prototype for the function would be:
\begin{verbatim}
char *int_to_string(int i);
\end{verbatim}
If we just ``extern-C'' it, then we can certainly call the function
and pass it an integer.  But we can't really use the string
that we get out, because we've asserted that the return type
is not a string, but rather a (possibly-null) pointer to a single
character.  So, when we call \texttt{foo} below:
\begin{verbatim}
extern "C" char *int_to_string(int i);

void foo() {
  int i = 12345;
  printf(int_to_string(i));
}
\end{verbatim}
we'll only get ``\texttt{1}'' for the output instead of
``\texttt{12345}''.  

If we know that the function always returns a pointer to a buffer of 
some fixed constant size, say \texttt{MAX\_NUM\_STRING}, then
we can change the prototype to:
\begin{verbatim}
extern "C" char *{MAX_NUM_STRING} int_to_string(int i);
\end{verbatim}
and we'll get the right behavior.  However, this obviously isn't
going to work if the size of the buffer might be different for
different calls.  

Another solution is to somehow convert the ``C string'' to a ``Cyclone
string'' before handing it back to Cyclone.  This is fundamentally
an unsafe operation because we must rely upon the ``C string'' being
properly zero-terminated.  So, your best bet is to write a little
wrapper function in C which can convert the C string to a Cyclone
string and then use that as follows:
\begin{verbatim}
extern "C" char *int_to_string(int i);
extern "C" char ?Cstring_to_string(char *);

void foo() {
  int i = 12345;
  printf(Cstring_to_string(int_to_string(i)));
}
\end{verbatim}
Fortunately, the Cyclone runtime (\texttt{lib/runtime\_cyc.c}) 
provides the needed routine which looks as follows:
\begin{verbatim}
// struct definition for fat pointers
struct _tagged_arr {
  unsigned char *curr;  // current pointer
  unsigned char *base;  // base address of buffer
  unsigned char *last_plus_one; // last_plus_one - base = size
};

struct _tagged_arr Cstring_to_string(char *s) {
  struct _tagged_arr str;
  if (s == NULL) {
    // return Cyclone fat NULL
    str.base = str.curr = str.last_plus_one = NULL;
  }
  else {
    int sz = strlen(s)+1;  // calculate string length + 1 for 0
    str.base = (char *)GC_malloc_atomic(sz); // malloc a new buffer
    if (str.base == NULL) // check that mallloc succeeded
      _throw_badalloc();
    str.curr = str.base;  // set current to base
    str.last_plus_one = str.base + sz; // set the size

    // Copy the string in case the C code frees it or mangles it
    str.curr[--sz] = '\0';
    while(--sz>=0)
      str.curr[sz]=s[sz];
  }
  return str; // return the fat pointer
}
\end{verbatim}
The \texttt{\_tagged\_arr} definition defines the struct type that
Cyclone uses to represent all fat pointers.  (It's actually defined
in a header file that gets included.)  Fat pointers are represented
using a ``current pointer'' which is the real pointer, and two
other pointers which represent the base address and maximum address
(plus one) for the buffer of objects.  

The second definition defines our wrapper function which returns a fat
pointer (\texttt{struct \_tagged\_arr}) given a C string.  You'll
notice that the function is bullet-proofed to avoid a number of
issues.  For instance, we first check to see if the C string is
actually \texttt{NULL} and if so, return a fat \texttt{NULL} 
(a struct where \texttt{curr}, \texttt{base}, and \texttt{last\_plus\_one}
are all \texttt{NULL}.)  If the C string is not \texttt{NULL},
we allocate a new buffer and copy the string over to the buffer.
This ensures that if C re-uses the storage (or frees it), Cyclone
won't get confused.  Notice also that we call \texttt{GC\_malloc\_atomic}
to allocate the storage.  In this case, we can use the atomic
malloc because we know the data do not contain pointers.
After copying the string, we initialize a \texttt{struct \_tagged\_arr}
appropriately and then return the struct.  

If we could ensure that the storage passed back to us wasn't
going to get recycled, then we could avoid the copy and simplify
the code greatly:
\begin{verbatim}
struct _tagged_arr Cstring_to_string(char *s) {
  struct _tagged_arr str;
  if (s == NULL) {
    // return Cyclone fat NULL
    str.base = str.curr = str.last_plus_one = NULL;
  }
  else {
    int sz = strlen(s)+1;  // calculate string length + 1 for 0
    str.base = str.curr = s; 
    str.last_plus_one = str.base + sz; // set the size
  }
  return str; // return the fat pointer
}
\end{verbatim}
Of course, using this is a bit more risky.  It's up to you
to make sure that you get the code right.

In porting various C libraries to Cyclone, we have had to write
a number of wrappers.  Doing so is fraught with peril and in the
future, we hope to provide tools that make this task easier
and easier to get right.  If you are planning to interface to
C code and need to write interfaces or wrappers, we encourage
you to look through the libraries to see how we have done
things.  

A particularly good example is the standard I/O library.  The
interface \texttt{lib/stdio.h} just includes \texttt{lib/cstdio.h}
and opens up the \texttt{Std} namespace.  (This makes it easier
to port C code, but if you want to keep the namespace closed,
you can directly include \texttt{lib/cstdio.h}.)  The \texttt{cstdio.h}
file is adapted from the BSD and Gnu \texttt{stdio.h} files
and shares a lot in common with them.  For instance, there
is an abstract struct for files, definitions for stdout, stdin,
stderr, various macros, and various function prototypes.

A typical example function is the one to remove a file which
has the following prototype:
\begin{verbatim}
extern int remove(const char ?);
\end{verbatim}
You'll notice that the function takes in a Cyclone string
as an argument.  Obviously, the ``real'' remove takes in a
C string.  What is going on here is that Cyclone defines
a wrapper function which, when given a Cyclone string,
converts it to a C string, and then calls C's remove. 

The wrapper function is defined in the file \texttt{stdio.cyc}.
Here are a few excerpts from that file:
\begin{verbatim}
namespace Cstdio {
  extern "C" {
    extern struct __sFILE;
    typedef struct Cstdio::__sFILE __sFILE;
    int remove(char *);
    int fclose(__sFILE);
    ...
  }
}

namespace Std;

abstract struct __sFILE {
  Cstdio::__sFILE *file; 
};

int remove(const char ? filename) {
  return Cstdio::remove(string_to_Cstring(filename));
}

int fclose(FILE @`r f) {
  if (f->file == NULL) return -1;
  int r = Cstdio::fclose((Cstdio::__sFILE @) f->file);
  if (r == 0) {
    f->file = NULL;
  }
  return r;
}

...
\end{verbatim}
At the top of the file, we have declared the external types
and functions that C uses.  Notice that these definitions
are wrapped in their own namespace (\texttt{Cstdio}) so that
we can ``redefine'' them within the \texttt{Std} namespace.
Also notice that they are wrapped with an extern-C so that
when compiled, their names won't get mangled.

The Cyclone wrapper code starts after the \texttt{namespace Std}
declaration.  The first thing we do is define a ``wrapper''
type for C files.  The wrapper includes a possibly null pointer
to a C file.  We use this level of indirection to keep someone
from closing a file twice, or from reading or writing to a file
that has been closed.  Of course, any operations on files will
need wrappers to strip off the level of indirection and check
that the file has not been closed already.  

The wrapper function for \texttt{remove} calls the \texttt{string\_to\_Cstring}
function (defined in \texttt{runtime\_cyc.c}) to conver the
argument to a C string and then passes the C string to the
real \texttt{remove} function, returning the error code.

The wrapper function for \texttt{fclose} checks to make sure that
the file has not already been closed.  If so, it returns -1.
Otherwise, it pulls out the real C file and passes it to the
real \texttt{fclose} function.  It then checks the return code
(to ensure that the close actually happened) and if it's 0,
sets the C file pointer to \texttt{NULL}, ensuring that we
don't call C's \texttt{fclose} on the file again.  

