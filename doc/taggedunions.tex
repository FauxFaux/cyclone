\section{Tagged Unions}
\hypertarget{tagged_unions}{}

In addition to \texttt{struct}, \texttt{enum}, and \texttt{union}, Cyclone
has \texttt{tunion} (for ``tagged union'') and \texttt{xtunion} (for
``extensible tagged union'') as ways to construct new aggregate types.
Like a \texttt{union} type, each \texttt{tunion} and \texttt{xtunion} has a
number of \textit{variants} (or members).  Unlike with \texttt{union},
an object of a \texttt{tunion} or \texttt{xtunion} type is exactly one
variant, we can detect (or discriminate) that variant at run-time, and
the language prevents using an object as though it had a different
variant.

The difference between \texttt{tunion} and \texttt{xtunion} is that
\texttt{tunion} is closed---a definition lists all possible variants.
It is like the algebraic datatypes in ML\@. With \texttt{xtunion},
separately compiled files can add variants, so no code can be sure
that it knows all the variants.  There is a rough analogy with not
knowing all the subclasses of a class in an object-oriented language.

For sake of specificity, we first explain how to
\hyperlink{tunion_sec}{create and use tunion} types.  We then
\hyperlink{xtunion_sec}{explain xtunion} by way of contrast with
\texttt{tunion}.  Because the only way to read parts of \texttt{tunion}
and \texttt{xtunion} types is pattern-matching, it is hard to understand
\texttt{tunion} without pattern-matching, but for sake of motivation and
completeness, some of the examples in the explanation of
pattern-matching use \texttt{tunion}!  To resolve this circular
dependency, we will informally explain pattern-matching as we use it
here and we stick to its simplest uses.

\subsection{tunion}\hypertarget{tunion_sec}{}

\paragraph{Basic Type Declarations and Subtyping}
\textit{[Warning: For expository purposes, this section contains a
  white lie that is exposed in the later section called ``regions for
  tunion''.]}

A \texttt{tunion} type declaration lists all of its variants.  At its
simplest, it looks just like an \texttt{enum} declaration.  For example,
we could say:

\begin{verbatim}
  tunion Color { Red, Green, Blue };
\end{verbatim}

As with \texttt{enum}, the declaration creates a type (called
\texttt{tunion Color}) and three constants \texttt{Red}, \texttt{Green}, and
\texttt{Blue}.  Unlike \texttt{enum}, these constants do not have type
\texttt{tunion Color}.  Instead, each variant has its \textit{own type},
namely \texttt{tunion Color.Red}, \texttt{tunion Color.Green}, and
\texttt{tunion Color.Blue}.  Fortunately these are all subtypes of
\texttt{tunion Color} and no explicit cast is necessary.  So you can
write, as expected:

\begin{verbatim}
  tunion Color c = Red;
\end{verbatim}

In this simple example, we are splitting hairs, but we will soon find
all these distinctions useful.  Unlike \texttt{enum}, \texttt{tunion}
variants may carry any fixed number of values, as in this example:

\begin{verbatim}
  tunion Shape {
    Point,
    Circle(float),
    Ellipse(float,float),
    Polygon(int,float),
  };
\end{verbatim}

A \texttt{Point} has no accompanying information, a \texttt{Circle} has a
radius, an \texttt{Ellipse} has two axis lengths, and a (regular)
\texttt{Polygon} has a number of sides and a radius.  (The value fields
do not have names, so it is often better style to have a variant carry
one value of a struct type, which of course has named members.)  This
example creates five types: \texttt{tunion Shape},
\texttt{tunion Shape.Point}, \texttt{tunion Shape.Circle},
\texttt{tunion Shape.Ellipse}, and \texttt{tunion Shape.Polygon}.  Like in
our previous example, \texttt{tunion Shape.Point} is a subtype of
\texttt{tunion Shape} and \texttt{Point} is a constant of
type \texttt{tunion Shape.Point}.

Variants that carry one or more values are treated differently.
\texttt{Circle} becomes a \textit{constructor}; given a float it produces an
object of type \texttt{tunion Shape.Circle}, for example \texttt{Circle(3.0)}.
Similarly, \texttt{Ellipse(0,0)} has type \texttt{tunion Shape.Ellipse}
(thanks to implicit casts from \texttt{int} to \texttt{float} for 0) and
\texttt{Polygon(7,4.0)} has type \texttt{tunion Shape.Polygon}.  The
arguments to a constructor can be arbitrary expressions of the correct
type, for example, \texttt{Ellipse(rand(), sqrt(rand()))}.

The second difference is that value-carrying variant types (e.g.,
\texttt{tunion Shape.Circle}) are not subtypes or the \texttt{tunion} type
(e.g., \texttt{tunion Shape}).  Rather \textit{non-null pointers} to the
value-carrying variant types are (e.g., \texttt{tunion Shape.Circle @}
is a subtype of \texttt{tunion Shape}).  So the following are correct
initializations that use implicit subtyping:

\begin{verbatim}
  tunion Shape s1 = Point;
  tunion Shape s2 = new Circle(3.0);
\end{verbatim}

\texttt{tunion} types are particularly useful for building recursive
structures.  For example, a small language of arithmetic expressions
might look like this:
\begin{verbatim}
  enum Unops { Negate, Invert};
  enum Binops { Add, Subtract, Multiply, Divide };
  tunion Exp {
    Int(int),
    Float(float),
    Unop(enum Unops, tunion Exp),
    Binop(enum Binops, tunion Exp, tunion Exp)
  };
\end{verbatim}

A function returning an expression representing the multiplication of
its parameter by two could like this:
\begin{verbatim}
  tunion Exp double_exp(tunion Exp e) {
    return new Binop(Multiply, new Int(2));
  }
\end{verbatim}

\paragraph{Accessing tunion Variants} Given a value of a \texttt{tunion}
type, such as \texttt{tunion Shape}, we do not know which variant it is.

For non-value variants, we can use a standard comparison.  Continuing
the example from above, ``\texttt{s1 == Point}'' would be true whereas
``\texttt{s2 == Point}'' would be false.

Analogous comparisons would not work for value-carrying variants
because these variants are pointers.  Rather than provide predicates
(perhaps of the form \texttt{isCircle(s1)}), Cyclone requires
pattern-matching.  For example, here is how you could define
\texttt{isCircle}:
\begin{verbatim}
  bool isCircle(tunion Shape s) {
    switch(s) {
    case &Circle(r):
      return true;
    default: return false;
    }
  }
\end{verbatim}

When a \texttt{switch} statement's argument has a \texttt{tunion} type,
the cases describe variants.  One variant of \texttt{tunion Shape} is a
pointer to a \texttt{Circle}, which carries one value.  The
corresponding pattern has \texttt{\&} for the pointer, \texttt{Circle} for
the constructor name, and one identifier for each value carried by
\texttt{Circle}.  The identifiers are binding occurrences (declarations,
if you will), and the initial values are the values of the fields of
the \texttt{Circle} at which \texttt{s} points.  The scope is the extent
of the case clause.  Pattern-matching works for non-value variants
too, but there is no \texttt{\&} because they are not pointers.

Here is another example:

\textit{[The reader is asked to indulge compiler-writers who have
  forgotten basic geometry.]}
\begin{verbatim}
  extern area_of_ellipse(float,float);
  extern area_of_poly(int,float);
  float area(tunion Shape s) {
    int ans;
    switch(s) {
    case Point:
      ans = 0;
      break;
    case &Circle(r):
      ans = 3.14*r*r;
      break;
    case &Ellipse(r1,r2):
      ans = area_of_ellipse(r1,r2);
      break;
    case &Polygon(sides,r):
      ans = area_of_poly(sides,r);
      break;
    }
    return ans;
  }
\end{verbatim}

The cases are compared in order against \texttt{s}.  The following are
compile-time errors:
\begin{itemize}
\item It is possible that a member of the \texttt{tunion} type matches
  none of the cases.  Note that default matches everything.
\item A case is useless because it could only match if one of the
  earlier cases match.  For example, a default case at the end of the
  \texttt{switch} in area would be an error.
\end{itemize}

We emphasize that Cyclone has much richer pattern-matching support
than we have used here.

\paragraph{Implementation}
Non-value variants are translated to distinct small integers.  Because
they are small, they cannot be confused with pointers to
value-carrying variants.  Value-carrying variants have a distinct
integer tag field followed by fields for the values carried.  Hence
all values of a tunion type occupy one word, either with a small
number or with a pointer.

\paragraph{Regions for tunion} We have seen that non-null pointers to
value-carrying variants are subtypes of the \texttt{tunion} type.  For
example, \texttt{tunion Shape.Circle @} is a subtype of
\texttt{tunion Shape}.  Because \texttt{tunion Shape.Circle @} is a
pointer into the heap (it is shorthand for
\texttt{tunion Shape.Circle @`H}),
it would seem that all values of type \texttt{tunion Shape} are either
non-value variants or pointers into the heap.  In fact, this is true,
but only because \texttt{tunion Shape} is itself shorthand for
\texttt{tunion `H Shape}.

In other words, \texttt{tunion} types are region-polymorphic over the region
into which the value-carrying variants point.  An explicit region
annotation goes after \texttt{tunion}, just like an explicit region annotation
goes after \texttt{*} or \texttt{@}.  Here is an example using a stack
region:
\begin{verbatim}
  tunion Shape.Circle c = Circle(3.0);
  tunion _ Shape s = &c;
\end{verbatim}

The \texttt{_} is necessary because we did not give an explicit name to
the stack region.

We can now correct the white lie from the ``basic type declarations and
subtyping'' section.  A declaration \verb|tunion Foo {...}| creates a
type constructor which given a region creates a type.  For any region
\texttt{`r}, \texttt{tunion `r Foo} is a subtype of
\texttt{tunion Foo.Bar @`r} if \texttt{tunion Foo.Bar} carries values.  If
\texttt{tunion Foo.Bar} does not carry values, then it is a subtype of
\texttt{tunion `r Foo} for all \texttt{`r}.

\paragraph{Polymorphism and tunion} A \texttt{tunion} declaration may be
polymorphic over types and regions just like a \texttt{struct}
definition (see the section on
\hyperlink{polymorphism}{polymorphism}).  For example, here is a
declaration for binary trees where the leaves can hold some
\texttt{BoxKind} \texttt{`a}:
\begin{verbatim}
  tunion <`a> Tree {
    Leaf(`a);
    Node(tunion Tree<`a>, tunion Tree<`a>);
  };
\end{verbatim}

In the above example, the root may be in any region, but all children
will be in the heap.  This version allows the children to be in any
region, but they must all be in the same region.  (The root can still
be in a different region.)

\begin{verbatim}
  tunion <`a,`r::R> Tree {
    Leaf(`a);
    Node(tunion `r Tree<`a,`r>, tunion `r Tree<`a,`r>);
  };
\end{verbatim}

\paragraph{Existential Types} \textit{[This feature is independent of
  the rest of tunion's features and can be safely ignored when first
  learning Cyclone.]}

In addition to polymorphic \texttt{tunion} types, it is also possible to
parameterize individual variants by additional type variables.  (From
a type-theoretic point of view, these are existentially-quantified
variables.)  Here is a useless example:

\begin{verbatim}
  tunion T { Foo<`a>(`a, `a, int), Bar<`a,`b>(`a, `b), Baz(int) };
\end{verbatim}

The constructors for variants with existential types are used the same
way, for example \texttt{Foo("hi","mom",3)}, \texttt{Foo(8,9,3)}, and
\texttt{Bar("hello",17)} are all well-typed.  The compiler checks that
the type variables are used consistently---in our example, the first
two arguments to \texttt{Foo} must have the same type.  There is no need
(and currently no way) to explicitly specify the types being used.

Once a value of an existential variant is created, there is no way to
determine the types at which it was used.  For example,
\texttt{Foo("hi","mom",3)} and \texttt{Foo(8,9,3)} both have type, ``there
exists an `a such that the type is \texttt{Foo<`a>}''.  When
pattern-matching an existential variant, you must give an explicit
name to the type variables; the name can be different from the name in
the type definition.  Continuing our useless example, we can write:
\begin{verbatim}
  void f(tunion T t) {
    switch(t) {
    case Foo<`a>(x,y,z): return;
    case Bar<`b,`c>(x,y): return;
    case Baz(x): return;
    }
  }
\end{verbatim}

The scope of the type variables is the body of the case clause.  So in
the first clause we could create a local variable of type \texttt{`a}
and assign \texttt{x} or \texttt{y} to it.  Our example is fairly
``useless'' because there is no way for code to use the values of
existentially quantified types.  In other words, given
\texttt{Foo("hi","mom",3)}, no code will ever be able to use the strings
\texttt{"hi"} or \texttt{"mom"}.  Useful examples invariably use function
pointers.  For a realistic library, see fn.cyc in the distribution.
Here is a smaller (and sillier) example; see the section on region and
effects for an explanation of why the \texttt{`e} stuff is necessary.
\begin{verbatim}
  int f1(int x, int y) { return x+y; }
  int f2(string x, int y) {printf("%s",x); return y; }
  tunion T<`e::E> { Foo<`a>(`a, int f(`a, int; `e)); };
  void g(bool b) {
    tunion T<{}> t;
    if(b)
      t = Foo(37,f1);
    else
      t = Foo("hi",f2);
    switch(t) {
    case Foo<`a>(arg,fun):
      `a x = arg;
      int (*f)(`a,int;{}) = fun;
      f(arg,19);
      break;
    }
  }
\end{verbatim}

The case clause could have just been \texttt{fun(arg)}---the compiler
would figure out all the types for us.  Similarly, all of the explicit
types above are for sake of explanation; in practice, we tend to rely
heavily on type inference when using these advanced typing constructs.

\paragraph{Future}
\begin{itemize}
\item Currently, given a value of a variant type (e.g.,
  \texttt{tunion Shape.Circle}), the only way to access the fields is
  with pattern-matching even though the variant is known.  We may
  provide a tuple-like syntax in the future.
\item If a \texttt{tunion} has only one value-carrying variant, it does
  not need a tag field in its implementation.  We have not yet
  implemented this straightforward optimization.
\end{itemize}

\subsection{xtunion}\hypertarget{xtunion_sec}{}

We now explain how an \texttt{xtunion} type differs from a \texttt{tunion}
type.  The main difference is that later declarations may continue to
add variants.  Extensible datatypes are useful for allowing clients to
extend data structures in unforeseen ways.  For example:
\begin{verbatim}
  xtunion Food;
  xtunion Food { Banana; Grape; Pizza(list_t<xtunion Food>) };
  xtunion Food { Candy; Broccoli };
\end{verbatim}

After these declarations, \texttt{Pizza(new List(Broccoli, null))} is a
well-typed expression.

If multiple declarations include the same variants, the variants must
have the same declaration (the number of values, types for the values,
and the same existential type variables).

Because different files may add different variants and Cyclone
compiles files separately, no code can know (for sure) all the
variants of an \texttt{xtunion}. Hence all pattern-matches against a
value of an \texttt{xtunion} type must end with a case that matches
everything, typically \texttt{default}.

There is one built-in \texttt{xtunion} type: \texttt{xtunion exn} is the
type of exceptions.  Therefore, you declare new \texttt{xtunion exn}
types like this:

\begin{verbatim}
  xtunion exn {BadFilename(string)};
\end{verbatim}

The implementation of \texttt{xtunion} types if very similar to that of
\texttt{tunion} types, but non-value variants cannot be represented as
small integers because of separate compilation.  Instead, these
variants are represented as pointers to unique locations in static
data.  Creating a non-value variant still does not cause allocation.

% Local Variables:
% TeX-master: "main-screen"
% End:
