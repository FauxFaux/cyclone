\subsection{The C interface tool, \texttt{buildlib}}

\texttt{buildlib} is a tool that semi-automatically constructs a
Cyclone interface to C code.  It scans C header files and builds
Cyclone header files and stub code so that Cyclone programs can call
the C code.  We use it to build the Cyclone interface to the C
standard library (in much the same way that \texttt{gcc} uses the
\texttt{fixincludes} program).

To use \texttt{buildlib}, you must construct a \emph{spec file} that
tells it what C headers to scan, and what functions and constants to
extract from the headers.  By convention, the names of spec files end
in \texttt{.cys}.  If \texttt{spec.cys} is a spec file, then
\texttt{buildlib} is invoked by
\begin{verbatim}
  buildlib spec.cys
\end{verbatim}
The output of \texttt{buildlib} is placed in a directory,
\texttt{BUILDLIB.OUT}.  The output consists of Cyclone header files
and the stub files \texttt{cstubs.c} and \texttt{cycstubs.cyc}.  

\subsubsection*{Spec files}

The form of a spec file is given by the following grammar.

\begin{rules}{spec-file}
  (empty)\\
  \nt{spec} \nt{spec-file}
\end{rules}
\begin{rules}{spec}
  \nt{header-name} \tk{:} \nt{directives} \tk{;}
\end{rules}
\begin{rules}{directives}
  (empty)\\
  \nt{directive} \nt{directives}
\end{rules}
\begin{rules}{directive}
\tk{cpp} \tk{\lb} \nt{balanced-braces} \tk{\rb} \\
\tk{include} \tk{\lb} \nt{ids} \tk{\rb} \\
\tk{hstub} \nt{id}\opt \tk{\lb} \nt{balanced-braces} \tk{\rb} \\
\tk{cycstub} \nt{id}\opt \tk{\lb} \nt{balanced-braces} \tk{\rb} \\
\tk{cstub} \nt{id}\opt \tk{\lb} \nt{balanced-braces} \tk{\rb}
\end{rules}
\begin{rules}{ids}
  (empty)\\
  \nt{id} \nt{ids}
\end{rules}

The non-terminal \hypertarget{id}{\textit{id}} refers
to C identifiers, and \hypertarget{header-name}{\textit{header-name}}
ranges over C header names (e.g., \texttt{stdio.h},
\texttt{sys/types.h}).  We use
\hypertarget{balanced-braces}{\textit{balanced-braces}} to refer to any
sequence of C tokens with balanced braces, ignoring braces inside of
comments, strings, and character constants.

\subsubsection*{Directives}

\begin{description}
\item[include] The include directive is used to extract constants and
type definitions from C header files and put them into the equivalent
Cyclone header file.  For example, here is part of the spec that we
use to interface to C's \texttt{errno.h}:
\begin{verbatim}
  errno.h:
  include { E2BIG EACCES EADDRINUSE ... }
\end{verbatim}
The spec says that the Cyclone version of \texttt{errno.h} should use
the C definitions of error constants like \texttt{E2BIG}.  These are
typically macro-defined as integers, but the integers can differ from
system to system.  We ensure that Cyclone uses the right constants by
running buildlib on each system.

For another example, our spec for \texttt{sys/types.h} reads, in part:
\begin{verbatim}
  sys/types.h:
  include { id_t mode_t off_t pid_t ... }
\end{verbatim}
Here the symbols are typedef names, and the result will be that the
Cyclone header file contains the typedefs that define \texttt{id_t},
etc.  Again, these can differ from system to system.

You can use \texttt{include} to obtain not just constants (macros) and
typedefs, but struct and union definitions as well.  Furthermore, if a
definition you \texttt{include} requires any other definitions that
you do not explicitly \texttt{include}, those other definitions will
be placed into the Cyclone header too.

Currently, \texttt{include} does not work for variable or function
declarations.  You have to use the \texttt{hstub} directive to add
variable and function declarations to your Cyclone header.

\item[cstub]
The \texttt{cstub} directive adds code (the \nt{balanced-braces})
to the C stub file.  If an optional \nt{id} is used, then the code
will be added to the stub file only if the \nt{id} is declared by the
C header.  This is useful because every system defines a different
subset of the C standard library.

\item[cycstub]
The \texttt{cycstub} directive is like the \texttt{cstub} directive,
except that the code is added to the Cyclone stub file.

\item[hstub]
The \texttt{hstub} directive is like the \texttt{cstub} directive,
except that the code is added to the Cyclone header file.

\item[cpp]
The \texttt{cpp} directive is used to tell \texttt{buildlib} to scan
some extra header files before scanning the header file of the spec.
This is useful when a header file can't be parsed in isolation.  For
example, the standard C header \texttt{sys/resource.h} is supposed to
define \texttt{struct timeval}, but on some systems, this is defined
in \texttt{sys/types.h}, which must be included before
\texttt{sys/resource.h} for that file to parse.  This can be handled
with a spec like the following:
\begin{verbatim}
  sys/resource.h:
  cpp {
    #include <sys/types.h>
  }
  ...
\end{verbatim}

This will cause \texttt{sys/types.h} to be scanned by
\texttt{buildlib} before \texttt{sys/resource.h}.

You can also use the \texttt{cpp} directive to directly specify
anything that might appear in a C include file (e.g., macros).
\end{description}

\subsubsection*{Options}
\texttt{buildlib} has the following options.
\begin{description}
\item[-d \textit{directory}]
Use \textit{directory} as the output directory instead of the default
\texttt{BUILDLIB.OUT}.

\item[-gather and -finish]
\texttt{buildlib} works in two phases.  In the gather
phase, \texttt{buildlib} grabs the C headers listed in the spec file
from their normal locations in the C include tree, and stores them in
a special format in the output directory.  In the finish phase,
\texttt{buildlib} uses the specially formatted C headers to build the
Cyclone headers and stub files.  The \texttt{-gather} flag tells
\texttt{buildlib} to perform just the gather phase, and the
\texttt{-finish} flag tells it to perform just the finish phase.

\texttt{buildlib}'s two-phase strategy is intended to support cross
compilation.  A Cyclone compiler on one architecture can compile to a
second architecture provided it has the other architecture's Cyclone
header files.  These headers can be generated on the first
architecture from the output of the gather phase on the second
architecture.  This is more general than just having the second
architecture's Cyclone headers, because it permits works even in the
face of some changes in the spec file or \texttt{buildlib} itself
(which would change the other architecture's Cyclone headers).

\item[-gatherscript]
The \texttt{-gatherscript} flag tells \texttt{buildlib} to output a
shell script that, when executed, performs \texttt{buildlib}'s gather
phase.  This is useful when porting Cyclone to an unsupported
architecture, where \texttt{buildlib} itself does not yet work.  The
script can be executed on the unsupported architecture, and the result
can be moved to a supported architecture, which can then cross-compile
itself to the new architecture.

\end{description}

% Local Variables:
% TeX-master: "main-screen"
% End:
